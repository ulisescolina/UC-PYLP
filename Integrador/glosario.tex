% Terminos
\newglossaryentry{wearables}{
  name=wearables,
  description={\textit{Vestible}, En el contexto de la tecnología, hace referencia a un
  dispositivo que se pueda \textit{vestir}. Ej: relojes inteligentes}
  }

\newglossaryentry{dataset}{
  name=dataset,
  plural=datasets,
  description={\textit{Conjunto de datos}, sobre los cuales se realizan experimentos.
  Las conclusiones que los investigadores definan sobre un cierto tema, se da
  mediante los experimentos realizados sobre uno o más conjuntos de datos}
  }

\newglossaryentry{clickstream}{
  name=clickstream,
  description={\textit{Flujo de clicks}, es una bitacora detallada de como los usuarios
  navegan una página web al realizar una tarea}
  }

\newglossaryentry{continuum}{
  name=continuum,
  description={\textit{Continuo}, algo que cambia gradualmente o en pequeños
  incrementos sin ningún pico evidente}
  }

\newglossaryentry{pipelining}{
  name=pipelining,
  description={En Ciencias de la Computación, este hace
  referencia a una organización en la cual pasos sucesivos de una secuencias de
  instrucciones son ejecutadas por diferentes módulos, esto para que otra
  instrucción pueda iniciar antes que una instrucción anterior finalice}
  }

\newglossaryentry{framework}{
  name=framework,
  plural=frameworks,
  description={\textit{Marco de trabajo}, es un conjunto estandarizado de 
  conceptos, prácticas y criterios para enfocar un tipo de problemática particular
  que sirve como referencia, para enfrentar y resolver nuevos problemas de índole 
  similar}
  }

\newglossaryentry{cluster}{
  name=cluster,
  plural=clusters, 
  description={\textit{Grupo} o también llamado \textit{Granja de servidores},
  es un término que se aplica a los sistemas distribuidos y hace referencia a 
  un conjunto de máquinas interconectadas por una red de alta velocidad}
  }

\newglossaryentry{heartbeat}{
  name=heartbeat,
  plural=heartbeats, 
  description={\textit{Latido}, es una señal periódica generada por
  software para indicar que el funcionamiento está funcionando adecuadamente, o
  para la sincronización con otras partes del sistema}
  }

% Siglas
\newacronym{iot}{IoT}{Internet of Things}
\newacronym{bi}{BI}{Business Intelligence}
\newacronym{dbms}{DBMS}{Data Base Management Systems}
\newacronym{dw}{DW}{Data Warehouse}
\newacronym{edw}{EDW}{Enterprise Data Warehouse}
\newacronym{gfs}{GFS}{The Google File System}
\newacronym{hdfs}{HDFS}{Hadoop File System}
\newacronym{indec}{INDEC}{Instituto Nacional De Estadística y Censos}
\newacronym{mr}{MR}{MapReduce}
\newacronym{tb}{TB}{Terabyte}
\newacronym{pb}{PB}{Petabyte}
\newacronym{pram}{PRAM}{Parallel Random-Acces Machine}
\newacronym{zb}{ZB}{Zettabyte}
\newacronym{eb}{EB}{Exabyte}

