% Terminos
\newglossaryentry{wearables}{
  name=wearables,
  description={\textit{Vestible}, En el contexto de la tecnología, hace referencia a un
  dispositivo que se pueda \textit{vestir}. Ej: relojes inteligentes}
  }

\newglossaryentry{dataset}{
  name=dataset,
  description={\textit{Conjunto de datos}, sobre los cuales se realizan experimentos.
  Las conclusiones que los investigadores definan sobre un cierto tema, se da
  mediante los experimentos realizados sobre uno o más conjuntos de datos}
  }

\newglossaryentry{clickstream}{
  name=clickstream,
  description={\textit{Flujo de clicks}, es una bitacora detallada de como los usuarios
  navegan una página web al realizar una tarea}
  }

\newglossaryentry{continuum}{
  name=continuum,
  description={\textit{Continuo}, algo que cambia gradualmente o en pequeños
  incrementos sin ningún pico evidente}
  }

\newglossaryentry{pipelining}{
  name=pipelining,
  description={En Ciencias de la Computación, este hace
  referencia a una organización en la cual pasos sucesivos de una secuencias de
  instrucciones son ejecutadas por diferentes módulos, esto para que otra
  instrucción pueda iniciar antes que una instrucción anterior finalice}
  }

% Siglas
\newacronym{iot}{IoT}{Internet of Things}
\newacronym{bi}{BI}{Business Intelligence}
\newacronym{dbms}{DBMS}{Data Base Management Systems}
\newacronym{dw}{DW}{Data Warehouse}
\newacronym{edw}{EDW}{Enterprise Data Warehouse}
\newacronym{indec}{INDEC}{Instituto Nacional De Estadística y Censos}
\newacronym{tb}{TB}{Terabyte}
\newacronym{pb}{PB}{Petabyte}
\newacronym{pram}{PRAM}{Parallel Random-Acces Machine}
\newacronym{zb}{ZB}{Zettabyte}
\newacronym{eb}{EB}{Exabyte}

