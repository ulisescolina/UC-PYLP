\section{Pre-Requisitos}
\label{sec:prerequisitos}

Los pre-requisitos para poder realizar el tutorial que se describe en este
documento serán los siguientes:

\begin{enumerate}
\item {\bf GNU/Linux:} Es necesario tener instalada alguna distribución
    GNU/Linux, este es el sistema de-facto para desarrollo y producción
    de Hadoop, el mismo se probó en producción con clusters de más de
    2000 nodos, y aunque se puede instalar el mismo en una máquina con
    Windows\footnote{https://cwiki.apache.org/confluence/display/HADOOP2/Hadoop2OnWindows},
    este procedimiento no se cubre en el documento.
\item {\bf Java:} Hadoop requiere que el sistema tenga instalada una versión de
    Java, la versión de java dependerá de la versión de Hadoop, para descubrir cual
    versión es la recomendada, se puede ver las versiones
    compatibles\footnote{\url{https://cwiki.apache.org/confluence/display/HADOOP/Hadoop+Java+Versions}}.
    En este ejemplo no se está utilizando una versión oficial de Oracle, se está
    utilizando una versión libre, {\it OpenJDK 10.0.2}.
\item {\bf ssh:} Esta herramienta para el acceso remoto a equipos de una red
    determinada debe estar instalada para que la versión
    pseudo-distribuida y distribuida de Hadoop pueda establecer comunicación
    inter-nodos (o inter-procesos para la versión pseudo-distribuida).
\end{enumerate}
