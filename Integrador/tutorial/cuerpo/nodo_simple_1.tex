\section{Establecer el JAVA\_HOME}
\label{sec:establecer_java_home}

Como se menciona en el documento que describe al software
Hadoop\cite{ramirez2021}, este es una
herramienta construida en Java, y para su correcto funcionameiento se apoya
sobre una instalación de la \acrlong{jvm} (\acrshort{jvm}), la forma de
comunicar a Hadoop cuál es el directorio que contiene la instalación de la
\acrshort{jvm} es con la variable de entorno {\tt JAVA\_HOME}.

La variable de entorno del sistema que se denomina {\tt JAVA\_HOME} indica cual
es el directorio en el cual se encuentra instalada la versión de Java que está
utilizando la máquina. Para revisar si ya se tiene establecida la variable se
puede realizar lo siguiente: ``{\tt echo \$JAVA\_HOME}''.

Si el resultado de correr lo mencionado anteriormente es un directorio, el
sistema tiene establecida la variable en cuestión, de lo contrario será
necesario que se haga manualmente.

\subsection*{JAVA\_HOME}
Para establecer una nueva variable de entorno en el sistema es suficiente con
ejecutar 

\begin{lstlisting}[language=bash, caption=Exportar variable de entorno,
label=lst:exportar_java_home]
export JAVA_HOME="/usr/lib/jvm/java-10-openjdk/"
\end{lstlisting}

El valor que debe obtener la variable de entorno va a depender de que versión
de Java se tenga instalada además de si se tiene la versión que provee la
empresa Oracle, o si se tiene la versión libre.
