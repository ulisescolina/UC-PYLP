La instalación de Hadoop es sencilla de realizar, dependiendo de la conexión a
internet y el grado de cumplimiento de los pre-requisitos se puede tener
funcionando un ``\gls{cluster}'' de un solo nodo en cuestión de minutos,
primero vamos a recorrer los pasos necesarios para cumplimentar esto, un
\gls{cluster} de un solo nodo. Las versiones que posteriores a este documento
detallarán el procedimiento para la instalación del software en un entorno con
múltiples nodos.

\section{Obtención del paquete}
\label{sec:obtencion_del_paquete}

La primer tarea que tenemos que llevar a cabo es la obtención de Hadoop, para
esto se seguirán las recomendaciones propuestas por los desarrolladores de
descargar el archivo comprimido de la página oficial
\footnote{\url{http://www.apache.org/dyn/closer.cgi/hadoop/common/}}. Por el
momento se asume que el archivo descargado queda almacenado en: \\
$$\sim/Descargas/hadoop-3.3.1.tar.gz$$

\begin{tcolorbox}
  {\bf NOTA:} \\ \newline Otra de las formas que se puede obtener una
  instalación de Hadoop, es a través de los repositorios oficiales de la
  distribución que se encuentre ejecutando en la máquina. Los repositorios
  compatibles con Ubuntu tendrán la posibilidad de agregar un PPA y obtener los
  paquetes necesarios para Hadoop. \\ Más información acerca del repositorio
  disponible para Ubuntu y derivados (Mint, Debian\footnote{Ubuntu es una
  derivación de Debian, pero hay veces en donde los paquetes de una
  distribución no tienen problemas funcionando en la otra, sin embargo, esto no
  se probó en el caso específico de Hadoop.}) se pueden encontrar en el siguiente
  enlace:
  \\ \newline 
  \url{https://launchpad.net/~hadoop-ubuntu/+archive/ubuntu/stable} 
  \\ \newline Otras
  distribuciones (Fedora, RHEL, Manjaro, Gentoo, etc.) deberán buscar en
  sus repositorios oficiales en caso de querer seguir este acercamiento.
\end{tcolorbox}

\section{Configuración del entorno}
\label{sec:configuracion_del_entorno}

Aquí se describen cuestiones que son necesarias con los pre-requisitos, pero
que tienen que ver más con cuestiones de configuración del Sistema en general.
Estas cuestiones son de especial importancia si se quiere establecer el entorno
de acuerdo a las buenas prácticas para la implantación del \gls{cluster} con
múltiples nodos que actuen como {\it workers}, en este apartado de
configuración de entorno tendría que procederse a la creación de usuarios específicos
para el uso del cluster, la generación de claves asíncronas para la conexión con los
demás nodos del cluster. 

Sin embargo, se omiten estas tareas para poder retomarlas en una versión futura del
documento, primero nos ocuparemos de tener una versión funcional de Hadoop en
un nodo.

% \subsection*{Generación de usuario para Hadoop, SSH}

% Este paso es opcional y puede ser omitido por completo si simplemente se
% pretende utilizar la instalación en una configuración de nodo simple, sin
% embargo, todo es parte de las buenas prácticas a seguir a la hora de la configuración
% de este entorno.

% Nuevamente, acá será necesario ir al manual de cada distribución para saber
% cual es la herramienta que se tiene disponible para indicarle al sistema la
% creación de un nuevo usuario y un nuevo grupo. Acá un ejemplo con las
% herramientas que brinda el Ubuntu.

% \begin{lstlisting}[language=bash, caption=Creación de usuario y grupo para
% hadoop, label=lst:usuario_y_grupo_hadoop]
% sudo addgroup hadoop  
% sudo adduser -m --ingroup hadoop hduser
% \end{lstlisting}

% Con lo visto en el listing \ref{lst:usuario_y_grupo_hadoop} se crea un grupo
% que se denomina {\tt hadoop}, y se crea un usuario {\tt hduser} que se agrega a
% ese grupo recién creado. {\bf Importante:} Si se realiza esta parte, se puede 
% hacer todo teniendo en cuenta al directorio base como {\tt /home/hduser/}, si
% este paso se ignora se puede realizar el procedimiento desde el usuario por
% defecto sin problemas.

% Con esto hecho, se procede a cofigurar el servidor SSH que se tiene instalado,
% para esto se realiza lo siguiente:

% \begin{lstlisting}[language=bash, caption=Configuración de SSH,
% label=lst:config_ssh]
% su - hduser   
% ssh-keygen -t rsa -P ""
% cat .ssh/id_rsa.pub >> .ssh/authorized_keys
% \end{lstlisting}

% Al realizar esto, se puede probar acceder a la máquina a través de SSH con: \\
% {\tt ssh localhost} .

% Ahora se tiene que permitir al usuario creado al inicio de la sección tener
% permisos de superusuario, para esto se agrega los siguiente al final del archivo {\tt
% /etc/sudoers} (es necesario abrir este archivo con permisos de superusuario
% para poder editarlo): 

% \begin{lstlisting}[language=bash, caption=Permisos de superusuario para hduser,
% label=lst:superusuario_hduser]
% hduser ALL=(ALL:ALL) ALL
% \end{lstlisting}



