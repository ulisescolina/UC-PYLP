\documentclass[11pt]{diazessay} % Font size (can be 10pt, 11pt or 12pt)
% Ecuaciones
\usepackage{amsmath}
% Graficos
\usepackage{tikz}
% Bordes
\usepackage{tcolorbox}
% Setup bibliography file
\usepackage[
  sorting=none,
  backend=biber,
  style=ieee,
]{biblatex}

\addbibresource{bibliografia.bib}
% Referencias cliqueables
\usepackage{hyperref}
\hypersetup{colorlinks=false}

% Glosario
\usepackage[acronym]{glossaries}
\makeglossaries

% csquotes
\usepackage{csquotes}

% \usepackage{listings}
% \usepackage{minted}
% \usemintedstyle{default}
% \definecolor{codebg}{rgb}{0.96,0.96,0.96}
% -------------------------------

%---- Title Section -----
\title{
  \textbf{Apache Hadoop: Una guía paso a paso
} \\
  {\Large\itshapeUna guía paso a paso
}
}
 % Title and subtitle

\author{
  \textbf{Ramirez Ulises
} \\
  \textit{Universidad Nacional de Misiones
}\\
  \small\ttulisesrcolina@gmail.com

} % Author and institution

\date{\today} % Date, use \date{} for no date

%--------------------------------

\begin{document}
\section*{Historial de Versiones}
Acá se van a detallar los diferentes cambios que se realicen al trabajo luego
de cada entrega a fin de tener una mejor trazabilidad.

En el siguiente repositorio es el que contiene el versionado del trabajo:
\begin{center}
\url{https://github.com/ulisescolina/UC-PYLP/tree/master/Integrador}
\end{center}

\begin{center}
  \begin{tabular}{||p{2cm} p{10cm}||} 
    \hline
    Versión & Cambios \\ [0.5ex] 
    \hline\hline
    1.0.0 & Primer entrega. \\ [0.5ex]
    \hline
    1.0.1 & Segunda entrega (correcciones realizadas por la cátedra):
    \begin{itemize}
        \item Se corrige la forma de redacción en varias partes del documento.
        \item Se corrigen errores ortográficos.
        \item Se corrigen los acrónimos y se agrega la mención de su
          significado en la primer ocurrencia.
    \end{itemize} \\ %[0.5ex]
    \hline
  \end{tabular}
\end{center}
\newpage

 % Notas con respecto al historial de versiones
\maketitle % Print the title section

\renewcommand{\abstractname}{Resumen} % Cambiamos el titulo, en vez de
% decir 'Abstract' va a pasar a decir 'Resumen'
\begin{abstract}
  Este documento es una continuación de lo charlado en el documento introductorio
al modelo \acrlong{mr} (\acrshort{mr})~\cite{ramirez2021}, aquí se procede a la
descripción de cómo configurar un cluster Hadoop para el procesamiento paralelo
y distribuido. En esta instancia la configuración se hará sobre un único nodo.

Algo imporntante a tener en cuenta es que los pasos seguidos para esta configuración
pueden quedar obsoletos ante cambios en distintos paquetes de los cuales
depende el presente tutorial. Se recomienda que ante la incapacidad de poder
realizar una tarea se revisen los canales de distribución oficiales para el
paquete en cuestión.


  \textit{\textbf{Palabras clave:} Multiproceso, programación concurrente,
  Procesamiento distribuido, Big Data, MapReduce, Apache Hadoop.
} 
\end{abstract}
\vspace{30pt} % Vertical whitespace between the abstract and first section

%--- Cuerpo ---
\section{Introducción}
\label{sec:introduccion}

El comportamiento por defecto que tiene Hadoop le permite funcionar en entornos
que solamente cuentan con un nodo, a esto se lo llama ``modo local'' (o también
se lo denomina {\it standalone}), un nodo configurado en este modo permitirá la
compilación/ejecución de los mappers y los reducers. A lo largo de las
secciones siguientes se procederá con la configuración de un nodo Hadoop en modo
local.

% una vez se tiene una serie de nodos Hadoop
% configurados en modo standalone, es posible modificar un conjunto de archivos
% de configuración para permitir que estas máquinas interconectadas trabajen de
% manera conjunta.

\section{Pre-Requisitos}
\label{sec:prerequisitos}

Los pre-requisitos para poder realizar seguir el paso a paso serán los
siguientes:

\begin{enumerate}
\item {\bf GNU/Linux:} Es necesario tener instalada alguna distribución
GNU/Linux, este es el sistema de-facto para desarrollo y producción
de Hadoop, el mismo se probó en producción con clusters de mas de
2000 nodos, y aunque se puede instalar el mismo en una máquina con
Windows\footnote{https://cwiki.apache.org/confluence/display/HADOOP2/Hadoop2OnWindows},
este procedimiento no se cubre en el documento.
\item {\bf Java:} Hadoop requiere que el sistema tenga instalada una versión de
Java, la versión de java dependerá de la versión de Hadoop, para descubrir cual
versión es la recomendada, se puede ver las versiones.
compatibles\footnote{\url{https://cwiki.apache.org/confluence/display/HADOOP/Hadoop+Java+Versions}}
\item {\bf ssh:} Esta herramienta para el acceso remoto a equipos de una red
determinada debe estar instalada para que la versión
pseudo-distribuida y distribuida de Hadoop pueda establecer comunicación
inter-nodos (o inter-procesos para la versión pseudo-distribuida).
\end{enumerate}

%--- Fin Cuerpo ---
%- Bibliografía -
\clearpage
% \printbibheading
% \printbibliography[type=article, heading=subbibliography, title={Articulos}]
% \printbibliography[type=book, heading=subbibliography, title={Libros}]
% \printbibliography[type=online, heading=subbibliography, title={Recursos en
% línea}]
\printbibliography
%----------------

%--- Glosario ---
% \clearpage
\clearpage
% Terminos
\newglossaryentry{wearables}{
  name=wearables,
  description={\textit{Vestible}, En el contexto de la tecnología, hace referencia a un
  dispositivo que se pueda \textit{vestir}. Ej: relojes inteligentes}
  }

\newglossaryentry{dataset}{
  name=dataset,
  plural=datasets,
  description={\textit{Conjunto de datos}, sobre los cuales se realizan experimentos.
  Las conclusiones que los investigadores definan sobre un cierto tema, se da
  mediante los experimentos realizados sobre uno o más conjuntos de datos}
  }

\newglossaryentry{clickstream}{
  name=clickstream,
  description={\textit{Flujo de clicks}, es una bitacora detallada de como los usuarios
  navegan una página web al realizar una tarea}
  }

\newglossaryentry{continuum}{
  name=continuum,
  description={\textit{Continuo}, algo que cambia gradualmente o en pequeños
  incrementos sin ningún pico evidente}
  }

\newglossaryentry{pipelining}{
  name=pipelining,
  description={En Ciencias de la Computación, este hace
  referencia a una organización en la cual pasos sucesivos de una secuencias de
  instrucciones son ejecutadas por diferentes módulos, esto para que otra
  instrucción pueda iniciar antes que una instrucción anterior finalice}
  }

\newglossaryentry{framework}{
  name=framework,
  plural=frameworks,
  description={\textit{Marco de trabajo}, es un conjunto estandarizado de 
  conceptos, prácticas y criterios para enfocar un tipo de problemática particular
  que sirve como referencia, para enfrentar y resolver nuevos problemas de índole 
  similar}
  }

\newglossaryentry{cluster}{
  name=cluster,
  plural=clusters, 
  description={\textit{Grupo} o también llamado \textit{Granja de servidores},
  es un término que se aplica a los sistemas distribuidos y hace referencia a 
  un conjunto de máquinas interconectadas por una red de alta velocidad}
  }

\newglossaryentry{heartbeat}{
  name=heartbeat,
  plural=heartbeats, 
  description={\textit{Latido}, es una señal periódica generada por
  software para indicar que el funcionamiento está funcionando adecuadamente, o
  para la sincronización con otras partes del sistema}
  }

% Siglas
\newacronym{iot}{IoT}{Internet of Things}
\newacronym{bi}{BI}{Business Intelligence}
\newacronym{dbms}{DBMS}{Data Base Management Systems}
\newacronym{dw}{DW}{Data Warehouse}
\newacronym{edw}{EDW}{Enterprise Data Warehouse}
\newacronym{eb}{EB}{Exabyte}
\newacronym{gb}{GB}{Gibabyte}
\newacronym{gfs}{GFS}{The Google File System}
\newacronym{hdfs}{HDFS}{Hadoop File System}
\newacronym{indec}{INDEC}{Instituto Nacional De Estadística y Censos}
\newacronym{jar}{JAR}{Java Archive Format}
\newacronym{jvm}{JVM}{Java Virtual Machine}
\newacronym{mr}{MR}{MapReduce}
\newacronym{tb}{TB}{Terabyte}
\newacronym{pb}{PB}{Petabyte}
\newacronym{pram}{PRAM}{Parallel Random-Acces Machine}
\newacronym{zb}{ZB}{Zettabyte}


\printglossary[type=\acronymtype] % Si no esta este no imprime los acrónimos
\printglossary % Si no esta este con el anterior no imprime los términos del glosario
%----------------
\end{document}
