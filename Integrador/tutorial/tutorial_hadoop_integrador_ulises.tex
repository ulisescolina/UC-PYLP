\documentclass[11pt]{diazessay} % Font size (can be 10pt, 11pt or 12pt)
% Ecuaciones
\usepackage{amsmath}
% Graficos
\usepackage{tikz}
% Bordes
\usepackage{tcolorbox}
% Setup bibliography file
\usepackage[
  sorting=none,
  backend=biber,
  style=ieee,
]{biblatex}

\addbibresource{bibliografia.bib}
% Referencias cliqueables
\usepackage{hyperref}
\hypersetup{colorlinks=false}

% Glosario
\usepackage[acronym]{glossaries}
\makeglossaries

% csquotes
\usepackage{csquotes}

\usepackage{listings}
\usepackage{xcolor}

\definecolor{codegreen}{rgb}{0,0.6,0}
\definecolor{codegray}{rgb}{0.5,0.5,0.5}
\definecolor{codepurple}{rgb}{0.58,0,0.82}
\definecolor{backcolour}{rgb}{0.95,0.95,0.92}

\lstdefinestyle{tutorialpylp}{
    backgroundcolor=\color{backcolour},
    commentstyle=\color{codegreen},
    keywordstyle=\color{magenta},
    numberstyle=\tiny\color{codegray},
    stringstyle=\color{codepurple},
    basicstyle=\ttfamily\footnotesize,
    breakatwhitespace=false,
    breaklines=true,
    captionpos=b,
    keepspaces=true,
    numbers=left,
    numbersep=5pt,
    showspaces=false,
    showstringspaces=false,
    showtabs=false,
    tabsize=2
}
\lstset{style=tutorialpylp}
% \usepackage{minted}
% \usemintedstyle{default}
% \definecolor{codebg}{rgb}{0.96,0.96,0.96}
% -------------------------------

%---- Title Section -----
\title{
  \textbf{Apache Hadoop: Una guía paso a paso
} \\
  {\Large\itshapeUna guía paso a paso
}
}
 % Title and subtitle

\author{
  \textbf{Ramirez Ulises
} \\
  \textit{Universidad Nacional de Misiones
}\\
  \small\ttulisesrcolina@gmail.com

} % Author and institution

\date{\today} % Date, use \date{} for no date

%--------------------------------

\begin{document}
\section*{Historial de Versiones}
Acá se van a detallar los diferentes cambios que se realicen al trabajo luego
de cada entrega a fin de tener una mejor trazabilidad.

En el siguiente repositorio es el que contiene el versionado del trabajo:
\begin{center}
\url{https://github.com/ulisescolina/UC-PYLP/tree/master/Integrador}
\end{center}

\begin{center}
  \begin{tabular}{||p{2cm} p{10cm}||} 
    \hline
    Versión & Cambios \\ [0.5ex] 
    \hline\hline
    1.0.0 & Primer entrega. \\ [0.5ex]
    \hline
    1.0.1 & Segunda entrega (correcciones realizadas por la cátedra):
    \begin{itemize}
        \item Se corrige la forma de redacción en varias partes del documento.
        \item Se corrigen errores ortográficos.
        \item Se corrigen los acrónimos y se agrega la mención de su
          significado en la primer ocurrencia.
    \end{itemize} \\ %[0.5ex]
    \hline
  \end{tabular}
\end{center}
\newpage

 % Notas con respecto al historial de versiones
\maketitle % Print the title section

\renewcommand{\abstractname}{Resumen} % Cambiamos el titulo, en vez de
% decir 'Abstract' va a pasar a decir 'Resumen'
\begin{abstract}
  Este documento es una continuación de lo charlado en el documento introductorio
al modelo \acrlong{mr} (\acrshort{mr})~\cite{ramirez2021}, aquí se procede a la
descripción de cómo configurar un cluster Hadoop para el procesamiento paralelo
y distribuido. En esta instancia la configuración se hará sobre un único nodo.

Algo imporntante a tener en cuenta es que los pasos seguidos para esta configuración
pueden quedar obsoletos ante cambios en distintos paquetes de los cuales
depende el presente tutorial. Se recomienda que ante la incapacidad de poder
realizar una tarea se revisen los canales de distribución oficiales para el
paquete en cuestión.


  \textit{\textbf{Palabras clave:} Multiproceso, programación concurrente,
  Procesamiento distribuido, Big Data, MapReduce, Apache Hadoop.
} 
\end{abstract}
\vspace{30pt} % Vertical whitespace between the abstract and first section

%--- Cuerpo ---
\section{Introducción}
\label{sec:introduccion}

El comportamiento por defecto que tiene Hadoop le permite funcionar en entornos
que solamente cuentan con un nodo, a esto se lo llama ``modo local'' (o también
se lo denomina {\it standalone}), un nodo configurado en este modo permitirá la
compilación/ejecución de los mappers y los reducers. A lo largo de las
secciones siguientes se procederá con la configuración de un nodo Hadoop en modo
local.

% una vez se tiene una serie de nodos Hadoop
% configurados en modo standalone, es posible modificar un conjunto de archivos
% de configuración para permitir que estas máquinas interconectadas trabajen de
% manera conjunta.

\section{Pre-Requisitos}
\label{sec:prerequisitos}

Los pre-requisitos para poder realizar seguir el paso a paso serán los
siguientes:

\begin{enumerate}
\item {\bf GNU/Linux:} Es necesario tener instalada alguna distribución
GNU/Linux, este es el sistema de-facto para desarrollo y producción
de Hadoop, el mismo se probó en producción con clusters de mas de
2000 nodos, y aunque se puede instalar el mismo en una máquina con
Windows\footnote{https://cwiki.apache.org/confluence/display/HADOOP2/Hadoop2OnWindows},
este procedimiento no se cubre en el documento.
\item {\bf Java:} Hadoop requiere que el sistema tenga instalada una versión de
Java, la versión de java dependerá de la versión de Hadoop, para descubrir cual
versión es la recomendada, se puede ver las versiones.
compatibles\footnote{\url{https://cwiki.apache.org/confluence/display/HADOOP/Hadoop+Java+Versions}}
\item {\bf ssh:} Esta herramienta para el acceso remoto a equipos de una red
determinada debe estar instalada para que la versión
pseudo-distribuida y distribuida de Hadoop pueda establecer comunicación
inter-nodos (o inter-procesos para la versión pseudo-distribuida).
\end{enumerate}

La instalación de Hadoop es sencilla de realizar, dependiendo de la conexión a
internet y el grado de cumplimiento de los pre-requisitos se puede tener
funcionando un ``\gls{cluster}'' de un solo nodo en cuestión de minutos,
primero vamos a recorrer los pasos necesarios para cumplimentar esto, un
\gls{cluster} de un solo nodo. Las versiones que posteriores a este documento
detallarán el procedimiento para la instalación del software en un entorno con
múltiples nodos.

\section{Obtención del paquete}
\label{sec:obtencion_del_paquete}

La primer tarea que tenemos que llevar a cabo es la obtención de Hadoop, para
esto se seguirán las recomendaciones propuestas por los desarrolladores de
descargar el archivo comprimido de la página oficial
\footnote{\url{http://www.apache.org/dyn/closer.cgi/hadoop/common/}}. Por el
momento se asume que el archivo descargado queda almacenado en: \\
$$\sim/Descargas/hadoop-3.3.1.tar.gz$$

\begin{tcolorbox}
  {\bf NOTA:} \\ \newline Otra de las formas que se puede obtener una
  instalación de Hadoop, es a través de los repositorios oficiales de la
  distribución que se encuentre ejecutando en la máquina. Los repositorios
  compatibles con Ubuntu tendrán la posibilidad de agregar un PPA y obtener los
  paquetes necesarios para Hadoop. \\ Más información acerca del repositorio
  disponible para Ubuntu y derivados (Mint, Debian\footnote{Ubuntu es una
  derivación de Debian, pero hay veces en donde los paquetes de una
  distribución no tienen problemas funcionando en la otra, sin embargo, esto no
  se probó en el caso específico de Hadoop.}) se pueden encontrar en el siguiente
  enlace:
  \\ \newline 
  \url{https://launchpad.net/~hadoop-ubuntu/+archive/ubuntu/stable} 
  \\ \newline Otras
  distribuciones (Fedora, RHEL, Manjaro, Gentoo, etc.) deberán buscar en
  sus repositorios oficiales en caso de querer seguir este acercamiento.
\end{tcolorbox}

\section{Configuración del entorno}
\label{sec:configuracion_del_entorno}

Aquí se describen cuestiones que son necesarias con los pre-requisitos, pero
que tienen que ver más con cuestiones de configuración del Sistema en general.
Estas cuestiones son de especial importancia si se quiere establecer el entorno
de acuerdo a las buenas prácticas para la implantación del \gls{cluster} con
múltiples nodos que actuen como {\it workers}, en este apartado de
configuración de entorno tendría que procederse a la creación de usuarios específicos
para el uso del cluster, la generación de claves asíncronas para la conexión con los
demás nodos del cluster. 

Sin embargo, se omiten estas tareas para poder retomarlas en una versión futura del
documento, primero nos ocuparemos de tener una versión funcional de Hadoop en
un nodo.

% \subsection*{Generación de usuario para Hadoop, SSH}

% Este paso es opcional y puede ser omitido por completo si simplemente se
% pretende utilizar la instalación en una configuración de nodo simple, sin
% embargo, todo es parte de las buenas prácticas a seguir a la hora de la configuración
% de este entorno.

% Nuevamente, acá será necesario ir al manual de cada distribución para saber
% cual es la herramienta que se tiene disponible para indicarle al sistema la
% creación de un nuevo usuario y un nuevo grupo. Acá un ejemplo con las
% herramientas que brinda el Ubuntu.

% \begin{lstlisting}[language=bash, caption=Creación de usuario y grupo para
% hadoop, label=lst:usuario_y_grupo_hadoop]
% sudo addgroup hadoop  
% sudo adduser -m --ingroup hadoop hduser
% \end{lstlisting}

% Con lo visto en el listing \ref{lst:usuario_y_grupo_hadoop} se crea un grupo
% que se denomina {\tt hadoop}, y se crea un usuario {\tt hduser} que se agrega a
% ese grupo recién creado. {\bf Importante:} Si se realiza esta parte, se puede 
% hacer todo teniendo en cuenta al directorio base como {\tt /home/hduser/}, si
% este paso se ignora se puede realizar el procedimiento desde el usuario por
% defecto sin problemas.

% Con esto hecho, se procede a cofigurar el servidor SSH que se tiene instalado,
% para esto se realiza lo siguiente:

% \begin{lstlisting}[language=bash, caption=Configuración de SSH,
% label=lst:config_ssh]
% su - hduser   
% ssh-keygen -t rsa -P ""
% cat .ssh/id_rsa.pub >> .ssh/authorized_keys
% \end{lstlisting}

% Al realizar esto, se puede probar acceder a la máquina a través de SSH con: \\
% {\tt ssh localhost} .

% Ahora se tiene que permitir al usuario creado al inicio de la sección tener
% permisos de superusuario, para esto se agrega los siguiente al final del archivo {\tt
% /etc/sudoers} (es necesario abrir este archivo con permisos de superusuario
% para poder editarlo): 

% \begin{lstlisting}[language=bash, caption=Permisos de superusuario para hduser,
% label=lst:superusuario_hduser]
% hduser ALL=(ALL:ALL) ALL
% \end{lstlisting}




\section{Establecer el JAVA\_HOME}
\label{sec:establecer_java_home}

Como se menciona en el documento que describe al software
Hadoop\cite{ramirez2021}, este es una
herramienta construida en Java, y para su correcto funcionameiento se apoya
sobre una instalación de la \acrlong{jvm} (\acrshort{jvm}), la forma de
comunicar a Hadoop cuál es el directorio que contiene la instalación de la
\acrshort{jvm} es con la variable de entorno {\tt JAVA\_HOME}.

La variable de entorno del sistema que se denomina {\tt JAVA\_HOME} indica cual
es el directorio en el cual se encuentra instalada la versión de Java que está
utilizando la máquina. Para revisar si ya se tiene establecida la variable se
puede realizar lo siguiente: ``{\tt echo \$JAVA\_HOME}''.

Si el resultado de correr lo mencionado anteriormente es un directorio, el
sistema tiene establecida la variable en cuestión, de lo contrario será
necesario que se haga manualmente.

\subsection*{JAVA\_HOME}
Para establecer una nueva variable de entorno en el sistema es suficiente con
ejecutar 

\begin{lstlisting}[language=bash, caption=Exportar variable de entorno,
label=lst:exportar_java_home]
export JAVA_HOME="/usr/lib/jvm/java-10-openjdk/"
\end{lstlisting}

El valor que debe obtener la variable de entorno va a depender de que versión
de Java se tenga instalada además de si se tiene la versión que provee la
empresa Oracle, o si se tiene la versión libre.

\section{Descomprimir el paquete descargado}
\label{sec:descomprimir_el_paquete_descargado}

Hasta ahora se realizaron configuraciones referentes al entorno en el cual va a
residir el nodo Hadoop, ahora se retoman las tareas relacionadas con el paquete 
descargado al inicio.

\begin{lstlisting}[language=bash, caption=Descompresión del paquete Hadoop, 
label=lst:descompresion_hadoop]
cd ~/Descargas/              # cambiar de directorio 
tar -xzf hadoop-3.1.1.tar.gz # descompresion del archivo
mv hadoop-3.3.1.tar/hadoop-3.1.1/ ~/hadoop/ 
                # se mueve todo el contenido a ~/hadoop/ 
\end{lstlisting}

Lo listado anteriormente describe como realizar la descompresión de lo
descargado en la sección \ref{sec:obtencion_del_paquete}, el primer paso es
cambiar de directorio hacia donde se encuenra el paquete comprimido, esto se
había asumido en la sección \ref{sec:obtencion_del_paquete} que es {\tt
~/Descargas/}, luego se descomprime con la herramienta {\tt tar} (para ver más
utilidades que proporciona la herramienta se puede ingresar {\tt man tar} en un
terminal), por último, se mueve el directorio y todo el contenido que resulta de
la descompresión al directorio {\tt ~/hadoop/}.

\section{Prueba de funcionamiento}
\label{sec:prueba_de_funcionamiento}

El directorio resultante de la descompresión contiene todo lo necesario para
ejecutar Hadoop, en este documento nos vamos a enfocar en los contenidos de
{\tt bin} y los contenidos de {\tt etc}. 

El directorio {bin} contiene todos los puntos de acceso para la herramienta, se
puede proceder a ejecutar lo siguiente para probar su funcionamiento:

\begin{lstlisting}[language=bash, caption=Prueba de funcionamiento, 
label=lst:prueba_de_funcionamiento]
cd ~/hadoop/
./bin/hadoop
\end{lstlisting}

Esto debería imprimir una pequeña ayuda acerca del paquete y es la prueba de
que tenemos funcionando una instalación de Hadoop en el ordenador.

\subsection*{El directorio {\tt etc}}

En este directorio se tienen archivos de configuración o scripts para la
modificación de valores por defecto utilizados dentro de Hadoop. Un ejemplo de
esto y que ya se vió anteriormente es el establecimiento de la variable de 
entorno {\tt JAVA\_HOME}, si se revisa el archivo {\tt hadoop-env.sh} se podrá
encontrar parte de lo escrito con anterioridad para establecer la variable
JAVA\_HOME.

Una de las notas en el documento que es de interés es la siguiente:

\begin{tcolorbox}
  {\tt
    Technically, the only required environment variable is JAVA\_HOME.
    All others are optional.  However, the defaults are probably not 
    preferred.  Many sites configure these options outside of Hadoop,
    such as in /etc/profile.d
  }
\end{tcolorbox}

Esta da la pauta de que la configuración adecuada puede ser parte de un
proceso que se tiene que adaptar a cada usuario, por lo que se tienen opciones
por defecto, sin embargo, estas pueden no ser las adecuadas. Para el ejercicio
que se realiza a lo largo del presente, se utilizan todos los valores por
defecto por el hecho de que se esta configurando un nodo simple a modo de
ejemplificación.

En versiones futuras del documento se planea continuar profundizando este asunto.

%--- Fin Cuerpo ---
%- Bibliografía -
\clearpage
% \printbibheading
% \printbibliography[type=article, heading=subbibliography, title={Articulos}]
% \printbibliography[type=book, heading=subbibliography, title={Libros}]
% \printbibliography[type=online, heading=subbibliography, title={Recursos en
% línea}]
\printbibliography
%----------------

%--- Glosario ---
% \clearpage
\clearpage
% Terminos
\newglossaryentry{wearables}{
  name=wearables,
  description={\textit{Vestible}, En el contexto de la tecnología, hace referencia a un
  dispositivo que se pueda \textit{vestir}. Ej: relojes inteligentes}
  }

\newglossaryentry{dataset}{
  name=dataset,
  plural=datasets,
  description={\textit{Conjunto de datos}, sobre los cuales se realizan experimentos.
  Las conclusiones que los investigadores definan sobre un cierto tema, se da
  mediante los experimentos realizados sobre uno o más conjuntos de datos}
  }

\newglossaryentry{clickstream}{
  name=clickstream,
  description={\textit{Flujo de clicks}, es una bitacora detallada de como los usuarios
  navegan una página web al realizar una tarea}
  }

\newglossaryentry{continuum}{
  name=continuum,
  description={\textit{Continuo}, algo que cambia gradualmente o en pequeños
  incrementos sin ningún pico evidente}
  }

\newglossaryentry{pipelining}{
  name=pipelining,
  description={En Ciencias de la Computación, este hace
  referencia a una organización en la cual pasos sucesivos de una secuencias de
  instrucciones son ejecutadas por diferentes módulos, esto para que otra
  instrucción pueda iniciar antes que una instrucción anterior finalice}
  }

\newglossaryentry{framework}{
  name=framework,
  plural=frameworks,
  description={\textit{Marco de trabajo}, es un conjunto estandarizado de 
  conceptos, prácticas y criterios para enfocar un tipo de problemática particular
  que sirve como referencia, para enfrentar y resolver nuevos problemas de índole 
  similar}
  }

\newglossaryentry{cluster}{
  name=cluster,
  plural=clusters, 
  description={\textit{Grupo} o también llamado \textit{Granja de servidores},
  es un término que se aplica a los sistemas distribuidos y hace referencia a 
  un conjunto de máquinas interconectadas por una red de alta velocidad}
  }

\newglossaryentry{heartbeat}{
  name=heartbeat,
  plural=heartbeats, 
  description={\textit{Latido}, es una señal periódica generada por
  software para indicar que el funcionamiento está funcionando adecuadamente, o
  para la sincronización con otras partes del sistema}
  }

% Siglas
\newacronym{iot}{IoT}{Internet of Things}
\newacronym{bi}{BI}{Business Intelligence}
\newacronym{dbms}{DBMS}{Data Base Management Systems}
\newacronym{dw}{DW}{Data Warehouse}
\newacronym{edw}{EDW}{Enterprise Data Warehouse}
\newacronym{eb}{EB}{Exabyte}
\newacronym{gb}{GB}{Gibabyte}
\newacronym{gfs}{GFS}{The Google File System}
\newacronym{hdfs}{HDFS}{Hadoop File System}
\newacronym{indec}{INDEC}{Instituto Nacional De Estadística y Censos}
\newacronym{jar}{JAR}{Java Archive Format}
\newacronym{jvm}{JVM}{Java Virtual Machine}
\newacronym{mr}{MR}{MapReduce}
\newacronym{tb}{TB}{Terabyte}
\newacronym{pb}{PB}{Petabyte}
\newacronym{pram}{PRAM}{Parallel Random-Acces Machine}
\newacronym{zb}{ZB}{Zettabyte}


\printglossary[type=\acronymtype] % Si no esta este no imprime los acrónimos
\printglossary % Si no esta este con el anterior no imprime los términos del glosario
%----------------
\end{document}
