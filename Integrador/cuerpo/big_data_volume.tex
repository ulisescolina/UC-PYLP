\subsection{Volúmen}
\label{ssec:volumen}

El volúmen hace referencia a la cantidad de datos dentro de un \gls{dataset}\cites{ghasemaghaei2019,
ghasemaghaei2021}, considerado por algunos autores como la característica que
brinda los mayores desafíos para el tratamiento con el big data\cite{che2013}.

Se habla de cantidad, y nuevamente se tiene esta idea de \textit{¿Qué tantos datos hacen
falta para tener un volúmen lo suficientemente masivo como para ser considerado
grande?}

Hacia el 2010, el mundo había creado 1\acrshort{zb}\footnote{NOTA: 1\acrshort{zb} equivale a $1 \times
10^{9}$ terabytes, es decir 1000000000\acrshort{tb}.} de datos, se menciona
esto para tener
una idea aproximada de la cantidad de datos manejados hace DIEZ AÑOS ATRAS! (una
eternidad en el mundo de las tecnologías de la
información),  y se estimaba que para el 2020 el mundo habría creado 40\acrshort{zb}\cite{lam2017}, 
sin embargo, para 2018 ya se tenían 33\acrshort{zb}, lo cual llevó al IDC a 
realizar nuevas consideraciones y se terminó estimando
que para 2025 se tendrían 175\acrshort{zb}\cite{forbes2020}. Es decir,
que en el tramo de 15 años, lo que se consideraba inconmensurable al inicio
multiplicaría su tamaño por 175. Hablar de estas magnitudes se
va naturalizando a medida que pasa el tiempo, ya en el 2013 se iba haciendo
cotidiano hablar de Petabytes (\acrshort{pb}) de cara al uso de
Exabytes (\acrshort{eb})\cite{che2013}. 

