\section{Introducción} \label{sec:introduccion}
% Se dice que se ganó interés en los datos y el volumen de generación de los
% mismos, como así tambien los desafíos que proponen.
El manejo de gran cantidad de datos ha ganado atención en los últimos
años\cite{newvantage2019},
esto surge a raíz del ritmo acelerado que se generan estos datos y los 
diferentes escenarios que surgen a la hora de asegurar\cites{rawat2019,
kantarcioglu2019}, procesar (lo que se discute en este documento, aunque este a
su vez se basa en muchos que van ser citados) y almacenar\cites{padgavankar2014, agrawal2016, hongming2017}
los mismos.


% Se relaciona la cantidad de personas en el mundo y sus actividades con el
% creciente volumen de generación de datos
En la actualidad existen mas de 7.8 mil millones de habitantes en el planeta\cite{world2021}, 
un porcentaje considerable de esos habitantes son usuarios de internet como
describe Meeker\cite{meeker2019}. Aunque es cierto que la generación de datos viene 
tomando una tendencia creciente desde hace ya años \cite{lyman2003},
parte de este crecimiento abrupto de la última decada se puede atribuir, en parte,
al gran papel que la tecnología
fué adquiriendo con el pasar de los años y su constante evolución: comercio electrónico, publicidad
electrónica, pagos electrónicos, etc.\cite{meeker2019,pellicer2019},
las personas conectadas con varios dispositivos (\gls{wearables}, dispositivos \acrshort{iot},
celulares «especial énfasis», televisores, etc.), sensores produciendo una
ráfaga continua de datos, el avance y la gran disponibilidad de las 
redes móviles de cara a las redes de siguiente generación
(5g)\cite{sultan2018}, dan como resultado una enorme cantidad de datos.

% Se introduce el término "Big Data"
Esta enorme cantidad de datos acuño un nombre con el pasar de los años, el
término es \textit{big data}.

Shönberger y Culkier\cite{shonberger2016} brindan una síntesis concreta de lo
que es big data, sin embargo, como los mismos autores lo mencionan,
esto no es nada mas que el inicio.

\begin{displayquote}
    ``Datos masivos, o cosas que se pueden hacer a gran
    escala, pero no a una escala inferior, con el fin de extraer nuevas
    percepciones o crear formas de valor de tal forma que se transformen los
    mercados.''
\end{displayquote}

Transformar estos datos en valor viene ya llevándose a cabo hace décadas,
caracterizado con términos que fueron cambiando con el tiempo, conmunmente denominado
\textit{inteligencia de negocios} (\acrshort{bi}) sobre el cual hay extensa literatura y
gran variedad de escenarios en los que se aplicó el mismo con diferentes
enfoques\cites{britos2008, qazi2014, su2021, khozium2020, vaclav2021, phillips-wren2021, schwade2021}.
El big data, toma estos principios de la inteligencia de negocios y los aplica a una escala mayor.

Aquí, se explora el procesamiento del big data, los desafíos
en cuanto a procesamiento que surgieron y siguen surgiendo (aunque
también se mencionan brevemente algunas cuestiones relacionadas con los
desafíos que intervienten con la seguridad y el almacenamiento de la
información), los enfoques a la hora de tratar de encontrar valor a una
cantidad de datos masivo teniendo en cuenta diferentes atributos con los que
cuentan. 

% \subsection{Problemática}
% \label{sec:problematica}

En la sección \ref{sec:big_data} se describe en mayor profundidad lo que
significa el big data y cuales son sus características predominantes, es decir,
cuando alguien considera algo como big data, en la sección \ref{sec:procesamiento_paralelo} se
introduce lo que es el procesamiento paralelo, se hace especial énfasis
en cómo éste permite el tratamiento de los datos haciendo uso eficiente
de todos los recursos. A partir de ahí, en la sección 
\ref{sec:procesamiento_distribuido} se aplica esta idea de utilización
eficiente de los recursos y se los traslada a
un entorno mayor no limitado a una máquina y con un nivel de acoplamiento
inferior.

