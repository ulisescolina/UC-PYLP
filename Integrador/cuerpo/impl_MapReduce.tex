\section{Hadoop}
\label{sec:hadoop}

\subsection*{Introducción}
Apache Hadoop es una herramienta que surge luego de dos grandes contribuciones
de las que se estuvieron hablando en las secciones anteriores, estas dos son,
el modelo de programación MapReduce~\cite{dean2004, dean2008}, y
\acrshort{gfs}~\cite{ghemawat2003}.

Este es un \gls{framework} para Big Data implementado en Java, esta
implementación consiste de dos capas principales, la primer capa es para
almacenamiento de datos, esta capa es
 un sistema de archivos distribuidos que está basado en \acrshort{gfs}, 
que se denomina \acrlong{hdfs} (\acrshort{hdfs}), la segunda capa, ataca el
problema del procesamiento de los datos y se denomina {\it Hadoop MapReduce
\Gls{framework}}~\cite{lee2012}.

Este \gls{framework} trabaja exclusivamente sobre pares clave valor ({\tt «K,
V»}), se puede resumir este funcionamiento de la siguiente manera~\cite{}:

\begin{tcolorbox}
  \centering
  {\tt input} <k1, k2>$\rightarrow$ {\tt mapper} $\rightarrow$ <k2, v2> $\rightarrow$
  {\tt reducer} $\rightarrow$ <k3, v3>
\end{tcolorbox}

\subsection*{Implementación: Mapper y Reducer}

Para demostrar de manera menos abstracta y teórica el funcionamiento del
framework, ahora se procede a la implementación de un {\it mapper} y un {\it
reducer} extremadamente sencillos. El propósito del código brindado en este 
ejemplo es el de contar
la cantidad de palabras en diferentes archivos distribuidos en el
\acrshort{hdfs}. En los listings \ref{listing:hadoop_mapper} y
\ref{listing:hadoop_reducer} se tiene el código respectivo para cada uno de estos. 

La implementación completa (no solamente del mapper y el reducer) se puede encontrar
en el Anexo: {\bf Implementación sobre HDFS}.

\begin{listing}[ht]
\inputminted[
  frame=lines,
  bgcolor=codebg,
  framesep=2mm,
  baselinestretch=1.2,
  fontsize=\footnotesize,
  linenos]{java}{cuerpo/codigo/TokenizerMapper.java}
\caption{Hadoop MapReduce: Mapper}
\label{listing:hadoop_mapper}
\end{listing}



Lo primero a notar con la implementación de la clase creada,
es el hecho de que extiende de {\tt Mapper}~\cite{apache_hadoop_mapper}. 

\mint{java}|extends Mapper<Object, Text, Text, IntWritable>|

La clase Mapper abstrae dentro del \gls{framework} la distribución de tareas a
los diferentes nodos en la etapa {\it Map}, esta clase se instancia una vez por
cada {\it InputSplit}\footnote{Representa el/los datos a ser procesados por un mapper
individual} \footnote{https://hadoop.apache.org/docs/r2.7.4/api/org/apache/hadoop/mapreduce/InputSplit.html}.

Lo siguiente a prestar atención, es la reimplementación del método {\tt map},
este método se aplica a cada registro dentro del InputSplit, la funcionalidad
por defecto es devolver el mismo registro.

\begin{minted}{java}
public void map(Object key, Text value, Context context) 
    throws IOException, InterruptedException {
  StringTokenizer itr = new StringTokenizer(value.toString());
  while (itr.hasMoreTokens()) {
    word.set(itr.nextToken());
    context.write(word, one);
  }
}
\end{minted}

Los primeros dos parametros reiteran lo mencionado anteriormente, el uso de clave-valor
{\tt (K, V)}. Aquí, {\tt Object key} representa a la clave del InputSplit
que le toca al mapper instanciado, y de la misma manera, {\tt Text value}
representa al dato del InputSplit asociado al mapper. Lo que queda pendiente es
el {\tt Context context}, este permite la comunicación con el resto del {\it sistema
distribuido}, este objeto contiene información referente a la configuarción del
trabajo actual, reporte de progreso, mensajes de estado a nivel de aplicación, 
\glspl{heartbeat}.

La información contenida en {\tt value} se transforma en un iterador, que
contiene en cada uno de sus elementos a una palabra (para este caso
particular), al recorrer este iterador, se escribe otro par {\tt (K2, V2)}, la
clave pasa a ser la palabra, y el contador pasa a ser un numero (1), en un
ejemplo concreto esto significa que, para el par {\tt (hadoop, 1)} la palabra
``hadoop'', tiene 1 repetición. Al terminar el recorrido se finaliza con la
``Emisión'' de estos pares (K2, V2) intermedios. Ahora la etapa {\it Reduce}
puede acceder a estos pares (K2, V2).

\begin{listing}[ht]
\inputminted[
  frame=lines,
  bgcolor=codebg,
  framesep=2mm,
  baselinestretch=1.2,
  fontsize=\footnotesize,
  linenos]{java}{cuerpo/codigo/IntSumReducer.java}
\caption{Hadoop MapReduce: Reducer}
\label{listing:hadoop_reducer}
\end{listing}



Al igual que en el mapper, acá se inicia prestando atención a la definición de
la clase {\it Reducer}, en este caso.

\mint{java}|extends Reducer<Text,IntWritable,Text,IntWritable> {|

Al igual que la etapa anterior, el reducer, acepta pares (K2,V2). La diferencia
es que estos pares, son el resultado de las operaciones realizadas por el 
{\it Mapper} en el listing \ref{listing:hadoop_mapper}.

\begin{minted}{java}
public void reduce(Text key, Iterable<IntWritable> values, Context context)
    throws IOException, InterruptedException {

  int sum = 0;
  for (IntWritable val : values) {
    sum += val.get();
  }
  result.set(sum);
  context.write(key, result);
}
\end{minted}

Nuevamente, aquí se cuenta con la clave {\tt Text key} y el valor\newline{\tt
Iterable<IntWritable> values}, este último, posee todas las instancias de una
clave específica. Para ilustrar este contenido de mejor manera, se presenta un
ejemplo, aquí se tiene el par,

\begin{tcolorbox}
\centering
{\tt (K, V) = (``hadoop'', Iterable [1, 1, 1, 1, 1])}
\end{tcolorbox}


Descomponiendo el ejemplo anterior se tiene lo siguiente, 

\begin{tcolorbox}
\centering
{\tt (``hadoop'', 1)} \\
{\tt (``hadoop'', 1)} \\
{\tt (``hadoop'', 1)} \\
{\tt (``hadoop'', 1)}
\end{tcolorbox}

Estos múltiples pares (K, V) resultaron de la etapa mapper (que se
recuerda que pueden estar en {\it archivos intermedios} separados en distintos
nodos), que se unen en la etapa número 5, explicada en la subsección \ref{sssec:reduce}
{\it Reduce}.

Resumiendo el significado puntual de lo visto en términos simples es,  

\begin{tcolorbox}
  Iterable<IntWritable> values = {\bf Todos los valores con la clave ``hadoop''}
\end{tcolorbox}

Recordando que los valores son siempre 1, estos son siempre el mismo valor
por el problema que se está atacando con el mapper en este ejemplo particular,
no necesariamente son {\bf siempre} 1 en todos los problemas.



