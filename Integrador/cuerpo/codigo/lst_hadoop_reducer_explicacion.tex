Al igual que en el mapper, acá se inicia prestando atención a la definición de
la clase {\it Reducer}, en este caso.

\mint{java}|extends Reducer<Text,IntWritable,Text,IntWritable> {|

Al igual que la etapa anterior, el reducer, acepta pares (K2,V2). La diferencia
es que estos pares, son el resultado de las operaciones realizadas por el 
{\it Mapper} en el listing \ref{listing:hadoop_mapper}.

\begin{minted}{java}
public void reduce(Text key, Iterable<IntWritable> values, Context context)
    throws IOException, InterruptedException {

  int sum = 0;
  for (IntWritable val : values) {
    sum += val.get();
  }
  result.set(sum);
  context.write(key, result);
}
\end{minted}

Nuevamente, aquí se cuenta con la clave {\tt Text key} y el valor\newline{\tt
Iterable<IntWritable> values}, este último, posee todas las instancias de una
clave específica. Para ilustrar este contenido de mejor manera, se presenta un
ejemplo, aquí se tiene el par,

\begin{tcolorbox}
\centering
{\tt (K, V) = (``hadoop'', Iterable [1, 1, 1, 1, 1])}
\end{tcolorbox}


Descomponiendo el ejemplo anterior se tiene lo siguiente, 

\begin{tcolorbox}
\centering
{\tt (``hadoop'', 1)} \\
{\tt (``hadoop'', 1)} \\
{\tt (``hadoop'', 1)} \\
{\tt (``hadoop'', 1)}
\end{tcolorbox}

Estos múltiples pares (K, V) resultaron de la etapa mapper (que se
recuerda que pueden estar en {\it archivos intermedios} separados en distintos
nodos), que se unen en la etapa número 5, explicada en la subsección \ref{sssec:reduce}
{\it Reduce}.

Resumiendo el significado puntual de lo visto en términos simples es,  

\begin{tcolorbox}
  Iterable<IntWritable> values = {\bf Todos los valores con la clave ``hadoop''}
\end{tcolorbox}

Recordando que los valores son siempre 1, estos son siempre el mismo valor
por el problema que se está atacando con el mapper en este ejemplo particular,
no necesariamente son {\bf siempre} 1 en todos los problemas.
