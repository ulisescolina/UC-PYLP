\subsection{Variedad}
\label{ssec:variedad}

Esta es otra de las características principales dentro del big data,
y hace referencia a la variedad de datos que conforman un \gls{dataset}. Este
fenómeno tiene lugar por el simple hecho de que existen casi ilimitadas fuentes
que pueden contribuir a un \gls{dataset}, el cual puede estar compuesto por
diferentes tipos y formas de representaciones\cite{che2013} lo cual hace al big
data grande y estos pueden ser resumidos en estructurados, semi-estructurados y
no-estructurados\cite{sagiroglu2013}.

Los datos estructurados y los semi estructurados caben en los Sistemas de Administración de Bases de
Datos (\acrshort{dbms}) y/o Data-Warehouses (\acrshort{dw}), acá se mencionan
los semi estructurados porque estos estan compuestos por archivos tales como
XML, en donde se tienen etiquetas para separar diferentes elementos de datos.
En cuanto a los datos no estructurados se puede mencionar a la web como una
fuente importante de los mismos, un ejemplo primordial aquí es el denominado
\gls{clickstream}\cites{russom2011, albert2010-c05}, aunque menciones
honorables van además para los mensajes de textos que se obtienen de las
compañias de celulares, datos generados por sensores en dispositivos
\acrshort{iot}, etc.
