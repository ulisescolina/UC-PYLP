\section{Big Data}
\label{sec:big_data}

Definir \textit{¿Qué es el big data?} es un trabajo no trivial, anteriormente, se brindó
un concepto que pretendía dar al lector un primer acercamiento al tema, sin
embargo este carece de completitud y especificidad. En
la literatura se demuestra que términos difusos como `grande' hacen aun más
difícil el determinar qué es y qué no es big data, esto supone tener que
definir variables que se van a tomar en cuenta en diferentes disciplinas con el
fin de poder determinar `¿Qué tan grande tiene que ser algo para ser grande?'.

La caracterización principal del big data y por lo tanto la respuesta a la
pregunta anterior se tiene en lo que la literatura llama \textit{las 3
Vs}\footnote{Existen autores que consideran que son
4\textit{V}s\cite{wibowo2019, ghasemaghaei2019}, 5\textit{V}s
\cites{hitzler2013, yin2015}, e incluso autores que
consideran más \textit{V}s~\cite{bonner2017}. Sin embargo, cabe
destacar que estas dos opiniones incluyen las 3\textit{V}s mencionadas
anteriormente.}\cites{bd_ibm, bd_aws, bd_oracle}. Antes de continuar con las \textit{V}s, es conveniente realizar una
aclaración importante: En algunos casos la definición o caracterización aceptada del big data
con las \textit{V}s \textit{no es aplicable}, por ejemplo, en el ámbito de la medicina, Baro {\it et. al}
\cite{baro2015} determinan que la cantidad de datos necesarios para ser grande
se satisface si se cumple que,


\begin{equation} \label{eq-baro2015-1}%
  \log_{10}(n * p) \geq 7 \\
\end{equation}

En donde se tiene que,

\begin{tabular}{l l}
$n$ & Cantidad de \textit{individuos estadísticos}\\
$p$ & Cantidad de variables a ser analizadas \\
    & dentro del \gls{dataset}\\
\end{tabular}

% Siendo $n$ la cantidad de \textit{individuos estadísticos}
% y $p$ la cantidad de variables a ser analizadas dentro del \gls{dataset}
% compuesto por un conjunto de artículos.


En este caso específico analiza el significado de ¿Qué es ser big data? para un
área en particular, \textit{la medicina}\footnote{cabe destacar que no es posible asumir que este
análisis llevado a cabo por los autores es trasladable a todas las
disciplinas}, aquí se busca enfocar y delimitar el concepto de
grande al área de la salud. El motivo principal detrás de esto es que luego de
estudiar los diferentes casos (entendiendo como \textit{casos} las
publicaciones en revistas científicas orientadas exclusivamente a la medicina)%a la hora de trabajar con los datos dentro de
en un ambiente orientado a la salud, no se satisfacen los criterios de
masividad\footnote{Comparado con la cantidad de consultas que recibe Google, la
cantidad de reseñas escritas en Amazon, la cantidad de comentarios que se
publican en Twitter, etc.}, pero aún así existe en el corpus material acerca
del big data asociado a la medicina, por lo que se hace útil poseer una
definición mas ajustada a dicha disciplina.


\subsection{Volúmen}
\label{ssec:volumen}

El volúmen hace referencia a la cantidad de datos dentro de un \gls{dataset}\cites{ghasemaghaei2019,
ghasemaghaei2021}, considerado por algunos autores como la característica que
brinda los mayores desafíos para el tratamiento con el big data\cite{che2013}.

Se habla de cantidad, y nuevamente se tiene esta idea de \textit{¿Qué tantos datos hacen
falta para tener un volúmen lo suficientemente masivo como para ser considerado
grande?}

Hacia el 2010, el mundo había creado 1\acrshort{zb}\footnote{NOTA: 1\acrshort{zb} equivale a $1 \times
10^{9}$ terabytes, es decir 1000000000\acrshort{tb}.} de datos, se menciona
esto para tener
una idea aproximada de la cantidad de datos manejados hace DIEZ AÑOS ATRAS! (una
eternidad en el mundo de las tecnologías de la
información),  y se estimaba que para el 2020 el mundo habría creado 40\acrshort{zb}\cite{lam2017}, 
sin embargo, para 2018 ya se tenían 33\acrshort{zb}, lo cual llevó al IDC a 
realizar nuevas consideraciones y se terminó estimando
que para 2025 se tendrían 175\acrshort{zb}\cite{forbes2020}. Es decir,
que en el tramo de 15 años, lo que se consideraba inconmensurable al inicio
multiplicaría su tamaño por 175. Hablar de estas magnitudes se
va naturalizando a medida que pasa el tiempo, ya en el 2013 se iba haciendo
cotidiano hablar de Petabytes (\acrshort{pb}) de cara al uso de
Exabytes (\acrshort{eb})\cite{che2013}. 


\subsection{Variedad}
\label{ssec:variedad}

Esta es otra de las características principales dentro del big data,
y hace referencia a la variedad de datos que conforman un \gls{dataset}. Este
fenómeno tiene lugar por el simple hecho de que existen casi ilimitadas fuentes
que pueden contribuir a un \gls{dataset}, el cual puede estar compuesto por
diferentes tipos y formas de representaciones\cite{che2013} lo cual hace al big
data grande y estos pueden ser resumidos en estructurados, semi-estructurados y
no-estructurados\cite{sagiroglu2013}.

Los datos estructurados y los semi estructurados caben en los Sistemas de Administración de Bases de
Datos (\acrshort{dbms}) y/o Data-Warehouses (\acrshort{dw}), acá se mencionan
los semi estructurados porque estos estan compuestos por archivos tales como
XML, en donde se tienen etiquetas para separar diferentes elementos de datos.
En cuanto a los datos no estructurados se puede mencionar a la web como una
fuente importante de los mismos, un ejemplo primordial aquí es el denominado
\gls{clickstream}\cites{russom2011, albert2010-c05}, aunque menciones
honorables van además para los mensajes de textos que se obtienen de las
compañias de celulares, datos generados por sensores en dispositivos
\acrshort{iot}, etc.

\subsection{Velocidad} \label{ssec:velocidad}

En la introducción se hizo mención de la utilidad de herramientas para
análisis de datos, más específicamente, se habló de Almacenes de Datos o 
\acrlong{dw}/\acrlong{edw} (\acrshort{dw}/\acrshort{edw}), y su contribución
superlativa a la Inteligencia de Negocios (\acrshort{bi}) mencionada en la
Sección \ref{sec:introduccion}.

Este sistema utilizado en \acrshort{bi} es una respuesta para múltiples
desafíos que se tienen dentro de una organización\cite{datalytics2}, en este
documento se hace énfasis concretamente en uno de los motivos, el 
\textit{para dar soporte a las desiciones que se toman a nivel estratégico}.
Para lograr ese soporte, la organización 
sigue una metodología que le va a permitir aplicar estas herramientas
analiticas con el fín de extraer el valor de los datos. La gente de
Datalytics\footnote{https://www.datalytics.com/}, menciona a lo largo de varias
charlas que componen el ciclo de ``DataSchool'', la importancia de estas
arquitecturas clásicas, sin embargo, el acercamiento clásico a la creciente
cantidad de datos brinda más desafíos, particularmente el que más resuena en el
contexto de este documento es el hecho de que esta arquitectura tradicional es 
\textit{lenta para reaccionar}\cite{datalytics3}.

El concepto de velocidad es esto, y es tan importante como los que se hablaron
anteriormente. La velocidad a la que se pueda realizar un análisis concreto y
tomar decisiones basadas en los resultados es clave. El análisis de un
fragmento de información acerca de la competencia, datos estadísticos realizados
por algún organismo ajeno a la organización que tiene una periodicidad anual
(ej: \acrshort{indec}), etc. posee una gran potencialidad, sin embargo, si no
se analiza oportunamente, el valor que pueda existir en esa información se
pierde.




% Alguna conclusión probablemente para el tema de BD
