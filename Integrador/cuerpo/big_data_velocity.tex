\subsection{Velocidad} \label{ssec:velocidad}

En la introducción se hizo mención de la utilidad de herramientas para
análisis de datos, más específicamente, se habló de Almacenes de Datos o 
\acrlong{dw}/\acrlong{edw} (\acrshort{dw}/\acrshort{edw}), y su contribución
superlativa a la Inteligencia de Negocios (\acrshort{bi}) mencionada en la
Sección \ref{sec:introduccion}.

Este sistema utilizado en \acrshort{bi} es una respuesta para múltiples
desafíos que se tienen dentro de una organización\cite{datalytics2}, en este
documento se hace énfasis concretamente en uno de los motivos, el 
\textit{para dar soporte a las desiciones que se toman a nivel estratégico}.
Para lograr ese soporte, la organización 
sigue una metodología que le va a permitir aplicar estas herramientas
analiticas con el fín de extraer el valor de los datos. La gente de
Datalytics\footnote{https://www.datalytics.com/}, menciona a lo largo de varias
charlas que componen el ciclo de ``DataSchool'', la importancia de estas
arquitecturas clásicas, sin embargo, el acercamiento clásico a la creciente
cantidad de datos brinda más desafíos, particularmente el que más resuena en el
contexto de este documento es el hecho de que esta arquitectura tradicional es 
\textit{lenta para reaccionar}\cite{datalytics3}.

El concepto de velocidad es esto, y es tan importante como los que se hablaron
anteriormente. La velocidad a la que se pueda realizar un análisis concreto y
tomar decisiones basadas en los resultados es clave. El análisis de un
fragmento de información acerca de la competencia, datos estadísticos realizados
por algún organismo ajeno a la organización que tiene una periodicidad anual
(ej: \acrshort{indec}), etc. posee una gran potencialidad, sin embargo, si no
se analiza oportunamente, el valor que pueda existir en esa información se
pierde.


