\section{Procesamiento Distribuido}
\label{sec:procesamiento_distribuido}

Este tiene muchas similitudes con los sistemas de Procesamiento Paralelo, es
más, es válido decir que los Sistemas Paralelos y los Sistemas Distribuidos son
un entrelazamiento de redes, que tienen la caracteristica de ser
continuas\footnote{
o un \gls{continuum} de redes.
\url{https://dictionary.cambridge.org/es/diccionario/ingles/continuum}}, lo que
significa que el parámetro que determina si el sistema es distribuido es el promedio de
distancias entre los nodos de procesamiento~\cite{hyde98}. El desafío que
propone Hyde, y también diferencia a los dos sistemas es el de poder determinar el
estado exacto de todo el sistema. Este es un desafío no-trivial, ya que dicho
estado no puede ser computado
instantaneamente por el hecho de que cada operación dentro de la red podría
requerir travesías por la misma.
