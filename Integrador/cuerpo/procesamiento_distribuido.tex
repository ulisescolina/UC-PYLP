\section{Procesamiento Distribuido}
\label{sec:procesamiento_distribuido}

Este tiene muchas similitudes con los sistemas de Procesamiento Paralelo, es
más, es válido decir que los Sistemas Paralelos y los Sistemas Distribuidos son
un entrelazamiento de redes, que tienen la caracteristica de ser
continuas\footnote{
o un \gls{continuum} de redes.
\url{https://dictionary.cambridge.org/es/diccionario/ingles/continuum}}, lo que
significa que el parámetro que determina si el sistema es distribuido es el promedio de
distancias entre los nodos de procesamiento~\cite{hyde98}. El desafío que
propone Hyde, y también diferencia a los dos sistemas es el de poder determinar el
estado exacto de todo el sistema. Este es un desafío no-trivial, ya que dicho
estado no puede ser computado
instantaneamente por el hecho de que cada operación dentro de la red podría
requerir travesías por la misma.

Esta caracteristica mencionada hace que sea muy complicado establecer una unica
definición de que significa un sistema distribuido. Para el propósito del
documento se hace útil una definición brindada por 
Tanenbaum \textit{et. al.}~\cite{tanenbaum2007a}, la cual sugiere que,

\begin{displayquote}
  ``Un sistema distribuido es una colección de computadoras independientes que
  parecen ser un único sistema coherente ante los usuarios.''
\end{displayquote}

Aquí hay dos cuestiones extremadamente importantes que son de interés para el
documento. La primera, se habla de una \textit{colección de computadoras} y se
hace énfasis en el hecho de que estos ordenadores son independientes, lo cual
llevan a otro aspecto a considerar para esta colección de máquinas, el hecho de
que son \textit{fácilmente escalables} y \textit{tolerante a fallos}. La
segunda, se habla de los usuarios ya sea que se esté hablando de \textit{personas
o programas} y como para estos es transparente el hecho de que están lidiando con
un sistma distribuido, es decir, estos tienen la sensación de utilizar un
sistema compuesto por un único ordenador.

Además de estas dos cuestiones que se hacen presentes en la definición
presentada, existen otros aspectos dentro de los sistemas distribuidos que
son importantes para el estudio de los mismos en profundidad, sin embargo, no es el
objetivo de este documento estudiar los sistemas distribuidos en profundidad,
a continuación se presenta un punto de referencia para que el
lector pueda seguir estudiando a los sistemas distribuidos y sus diferentes
aspectos, características y desafíos los cuales no son
pocos~\cite{tanenbaum1994, tanenbaum2007, siberschatz2013}.

\printbibliography[keyword={recomendacion_sistemas_distribuidos}, title={Referencias para 
el estudio de Sistemas Distribuidos}]
