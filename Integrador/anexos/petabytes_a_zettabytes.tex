\clearpage
\section*{Anexo: Terabytes $\Rightarrow$ Petabytes $\Rightarrow$ Exabytes $\Rightarrow$ Zettabytes}
\label{sec:petabytes_a_zettabytes}

A veces, tratar magnitudes tan extremas no permite transmitir de manera
comprensible lo masivo o lo ínfimo de tales magnitudes,
esto es frecuentemente el caso al hablar de cantidad de
datos tales como los Terabytes, Petabytes, Exabytes o Zettabytes.

Si el lector es una persona que esta familiarizada con los medios de almacenamiento
y su tamaño, resulta sencilla la explicación de que un Terabyte es mayor que
un Gigabyte, esto puede deberse a que el Gigabyte es una unidad que se utiliza
cotidianamente, por ejemplo, el tamaño de los pendrives, la memoria del celular,
el tamaño de discos en ordenadores, etc. Sin embargo, esta explicación puede
volverse un desafío si se intenta realizar la explicación a una persona que no
posee conocimiento alguno acerca de este tema, o incluso si la capacidad que se
intenta explicar es lo suficientemente masiva como por ejemplo los mencionados
Zettabytes (\acrshort{zb}).

En este anexo, se propone un ejemplo para que el lector (independientemente de
su nivel en el uso de tecnología) se familiarice y tenga una mejor
concepción la masividad a la hora de hablar del salto desde Gigabytes a
Zettabytes. En este ejemplo no se van a utilizar
conceptos de las ciencias de la computación, se va a tratar 
con segundos, minutos, horas, días y años, conceptos que se puede asumir son
familiares y que son asimilados por todos los lectores.

Para iniciar, podemos partir desde el Gigabyte (\acrshort{gb}), y podemos decir
que un \acrshort{gb} en este anexo va a contar como $1$ seg. Partiendo de esa
premisa, el siguiente cuadro posee las equivalencias en segundos para cada unidad
mencionada anteriormente, utilizando, minutos, horas, días y años a medida que
se haga necesario.

\begin{table}[h]
  \centering
  \begin{tabular}{||r r c c c c||} 
   \hline
    Unidad & Segundo(s) & Minutos & Horas & Días & Años \\ [0.5ex] 
   \hline\hline
   \acrlong{gb} (\acrshort{gb}) & 1 & --- & --- & --- & --- \\ 
   \hline
   \acrlong{tb} (\acrshort{tb}) & 1024 & 17 & --- & --- & --- \\
   \hline
   \acrlong{pb} (\acrshort{pb}) & 1048576 & --- & 291 & 12 & --- \\
   \hline
   \acrlong{eb} (\acrshort{eb}) & 1073741824 & --- & --- & 12428 & 34 \\
   \hline
   \acrlong{zb} (\acrshort{zb}) & 1099511627776 & --- & --- & --- & 34865 \\ [1ex] 
   \hline
  \end{tabular}
  \\
  \label{unidades:gb_a_zb}
  \caption{Escala de unidades con su equivalencia temporal}
\end{table}

Es decir, si se compara a el \acrlong{gb} con el \acrlong{zb}, se ve la
masividad de la que se está hablando, y se hace palpable al extrapolar estos
valores tan grandes a un concepto cotidiano como el tiempo. 1\acrshort{gb} = 1
segundo y tenemos que 1\acrshort{zb} = 34865 {\bf AÑOS}. Se puede quere acortar
la diferencia y comparar dos unidades más cercanas como por ejemplo el
\acrlong{pb} y el \acrlong{zb}. Pero incluso con este acercamiento la
diferencia es extrema ya que se tiene que 1\acrshort{pb} = 12 Días mientras que
1\acrshort{zb} es 34865 {\bf Años}.
