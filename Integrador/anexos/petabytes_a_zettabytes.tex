\clearpage
\section*{Anexo: Terabytes $\Rightarrow$ Petabytes $\Rightarrow$ Zettabytes}
\label{sec:petabytes_a_zettabytes}

A veces, tratar magnitudes tan extremas no permite transmitir de manera
comprensible lo masivo o lo ínfimo de tales magnitudes,
esto es frecuentemente el caso al hablar de cantidad de
datos tales como los Terabytes, Petabytes, Zettabytes o Exabytes.

Si el lector es una persona que esta familiarizada con los medios de almacenamiento
y su tamaño, resulta sencilla la explicación de que un Terabyte es varias veces mayor que
un Gigabyte, esto puede deberse a que el Gigabyte es una unidad que se utiliza
cotidianamente, por ejemplo, el tamaño de los pendrives, la memoria del celular,
el tamaño de discos en ordenadores, etc. Sin embargo, esta explicación puede
volverse un desafío si se intenta realizar la explicación a una persona que no
posee conocimiento alguno acerca de este tema.

En este anexo, se propone un ejemplo para que el lector (independientemente de
su nivel en el uso de tecnología) se familiarice y tenga una mejor
concepción la masividad a la hora de hablar del salto Petabytes y
Zettabytes\footnote{Este salto de magnitudes es equivalente a los saltos que se
dan del Byte, hasta el Gigabyte
(B$\rightarrow$KB$\rightarrow$MB$\rightarrow$GB). Cada salto es 1024 veces
mayor comparado a la magnitud anterior.}, en este ejemplo no se van a utilizar
conceptos de las ciencias de la computación, se va a tratar 
con segundos, minutos, horas, días y años, conceptos que se puede asumir son
familiares a todo lector.


