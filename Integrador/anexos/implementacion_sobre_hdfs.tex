\clearpage
\section*{Anexo: Implementación sobre HDFS}

En esta sección de anexo se presenta la implementación del ejemplo
explicado en la sección \ref{sec:hadoop} {\it Hadoop}. 

\paragraph{Que se cubre en este anexo:}
En este anexo se cubren,
las partes necesarias en el código además de las clases mapper y reducer
explicadas, los paquetes necesarios, la forma de compilación, la forma de
ejecutar el paquete compilado sobre un cluster y se incluyen además los
resultados de una demo realizada sobre un \gls{cluster}.

\paragraph{Que no se cubre en este anexo:}
En este anexo no se cubre la
instalación/configuración del \gls{cluster} Hadoop. Cabe mencionar que es
posible comprar poder de cómputo a proveedores como Amazon\footnote{\url{https://aws.amazon.com/es/emr/?whats-new-cards.sort-by=item.additionalFields.postDateTime&whats-new-cards.sort-order=desc}} o
Microsoft\footnote{\url{https://docs.microsoft.com/en-us/samples/azure-samples/durablefunctions-mapreduce-dotnet/big-data-processing-serverless-mapreduce-on-azure/}}
y tener funcionando un cluster Hadoop en minutos, sin ninguna configuración,
esperando a que se le adjunten trabajos a realizar y datos sobre los cuales
realizar los trabajos.

\subsection*{Partes necesarias además del Mapper y Reducer}
En los ejemplos, únicamente se mencionaron las clases Mapper y Reducer, estas
los objetos de estas clases son las que se distribuyen a los diferentes nodos
workers para que puedan procesar los datos, sin embargo, faltan más partes para
que este ejemplo pueda compilar. El ejemplo completo puede ser encontrado en
el repositorio que se encuentra en la siguiente nota al pie de
página\footnote{https://github.com/ulisescolina/UC-PYLP/tree/master/Integrador/cuerpo/codigo/hadoop-integrador}

\begin{listing}[ht]
\inputminted[
  frame=lines,
  bgcolor=codebg,
  framesep=2mm,
  baselinestretch=1.2,
  fontsize=\footnotesize,
  linenos]{java}{cuerpo/codigo/WordCount.java}
\caption{Clase envoltorio WordCount}
\label{listing:word_count}
\end{listing}


En el listing \ref{listing:word_count} se pueden notar dos cuestiones que
resaltan y que son completamente necesarias a la hora de construir un Mapper y
un Reducer para trabajar, no solamente sobre el \Gls{framework} Hadoop
MapReduce, sino que también algo necesario para los proyectos de Java.

Primero se atiende lo segundo, lo de un proyecto Java. Un proyecto Java siempre
tiene este punto de inicio llamado el método {\tt main()} que se encuentra
dentro de una clase objetivo para el compilador. En este caso, la clase
objetivo se define como {\tt WordCount}, dentro de esta se tienen a las
implementaciones del mapper y el reducer, y además se cuenta con el método
main que juega un rol importante en la primera parte del argumento de
{\it cuestiones necesarias para trabajar sobre el \gls{framework} Hadoop}.
Estas cuestiones necesarias tienen que ver con las acciones necesarias para
poder comunicarle al cluster las clases que extienden de Mapper y Reducer, la
creación de un ``trabajo'' y determinar de dónde sacar el input y hacia donde
mandar el output una vez terminado el trabajo, todo esto se implementa (en este
ejemplo en particular) en el método {\tt main()}.

\subsection*{Paquetes necesarios}
Se puede notar que en diferentes partes de los listings
\ref{listing:hadoop_mapper}, \ref{listing:hadoop_reducer} y
\ref{listing:word_count} se hace uso de diferentes clases que no se encuentran
definidas en ningun sitio en el código presentado, por ejemplo se tienen a las
clases más prominentes ya mencionadas, como {\tt Mapper} y {\tt Reducer}, pero
también notar que se tienen otras clases y tipos no nativos de Java que son
también implementaciones
de Hadoop, estos son {\tt Configuration}, {\tt Job}, {\tt FileInputFormat},
{\tt FileOutputFormat}, {\tt IntWritable}, etc. En el listing
\ref{listing:paquetes_a_incluir} se lista la forma sintácticamente correcta de
incluir los paquetes específicos de Hadoop.

\begin{listing}[ht]
\inputminted[
  frame=lines,
  bgcolor=codebg,
  framesep=2mm,
  baselinestretch=1.2,
  fontsize=\footnotesize,
  linenos]{java}{cuerpo/codigo/WordCount_paquetes.java}
\caption{Paquetes Hadoop que se necesitan incluir}
\label{listing:paquetes_a_incluir}
\end{listing}


\subsection*{Compilación}
Una vez que se tiene implementado el Mapper y el Reducer, se tiene definida la
clase que los envuelve y definida claramente las clases que se deben distribuir 
entre los {\it workers} para un trabajo determinado,
queda armar el paquete que en última instancia será ejecutado sobre el
\gls{framework} sobre el \gls{cluster}, en este caso, será armar un paquete
{\tt \acrshort{jar}}.

Acá se definen dos formas de realizar este procedimiento.

\subsubsection*{Desde Hadoop}
Hadoop, cuenta con las herramientas para realizar la compilación de un
proyecto, sin embargo este proceso nos dará unicamente el bytecode que puede
correr sobre una \acrlong{jvm} (\acrshort{jvm}), por lo que serán necesarios
dos pasos para la compilación exitosa de un proyecto.

\begin{minted}{bash}
\$ hadoop com.sun.tools.javac.Main WordCount.java
\end{minted}

Siendo {\tt WordCount.java} el archivo que contiene el proyecto. una vez
terminado este procedimiento, es necesario empaquetar los archivos resultantes
en un \acrshort{jar}, esto se logra de la siguiente manera.

\begin{minted}{bash}
\$ jar cf wc.jar WordCount*.class 
\end{minted}

Este último paso resultará en un archivo denominado {\tt wc.jar}. El cual
contendrá el proyecto compilado y listo para ser ejecutado sobre el
\gls{cluster}

\subsubsection*{Proyecto Maven}
Esta segunda opción utiliza un método de construcción ampliamente aceptado y
reconocido en la comunidad de desarrolladores Java, el proyecto 
Maven\footnote{https://maven.apache.org/}, esta es una herramienta acompaña al 
desarrollador Java a lo largo del ciclo de vida del software. Este ejemplo
asume que existe una instalación configurada de Maven.

En el ejemplo que se lista a continuación, se está utilizando un concepto dentro
de Maven llamado
{\it Archetype}\footnote{https://maven.apache.org/archetype/index.html}, lo
que, en pocas palabras, permite que se cree la estructura de un proyecto a partir de
una plantilla existente, y el resultado está listo para la compilación.

\begin{minted}{bash}
\$ mvn archetype:generate \
-DgroupId=com.pylp.integrador \
-DartifactId=hadoop-integrador \
-DarchetypeArtifactId=maven-archetype-quickstart \
-DinteractiveMode=false
\end{minted}

Este comando, crea la estructura de un proyecto Java listo para ser compilado.
El subdirectorio {\tt src} contiene el código necesario para el proyecto, y el
subdirectorio {\tt test} contiene los casos de prueba para el código en {\tt
src}. Se puede ver la implementación del código en el repositorio deo proyecto,
en este se realiaron algunas modificaciones necesarias al archivo {\tt pom.xml}
que actúa como archivo de configuración de la compilación y el empaquetado de
las clases. Las modificaciones se muestran en el siguiente listing

\begin{minted}{xml}
  <!-- ... -->
  <!-- Acá se tienen más cosas -->
  <!-- ... -->

  <packaging>jar</packaging>

  <!-- ... -->
  <!-- Acá se tienen más cosas -->
  <!-- ... -->
  <!-- Se importan los artefactos de Hadoop desde el repositorio
       de Maven -->
  <dependencies>
    <!-- https://mvnrepository.com/artifact/
                org.apache.hadoop/hadoop-common -->
    <dependency>
        <groupId>org.apache.hadoop</groupId>
        <artifactId>hadoop-common</artifactId>
        <version>2.8.0</version>
    </dependency>
    <!-- https://mvnrepository.com/artifact/
                org.apache.hadoop/hadoop-mapreduce-client-core -->
    <dependency>
        <groupId>org.apache.hadoop</groupId>
        <artifactId>hadoop-mapreduce-client-core</artifactId>
        <version>2.8.0</version>
    </dependency>
    <!-- https://mvnrepository.com/artifact/
                 org.apache.hadoop/hadoop-core -->
    <dependency>
      <groupId>org.apache.hadoop</groupId>
      <artifactId>hadoop-core</artifactId>
      <version>0.20.2</version>
    </dependency>
  </dependencies>

  <build>
    <pluginManagement>
      <plugins>
        <plugin>
          <artifactId>maven-compiler-plugin</artifactId>
          <version>3.8.0</version>
          <configuration>
            <source>1.6</source>
            <target>1.6</target>
          </configuration>
        </plugin>
      </plugins>
    </pluginManagement>
  </build>
\end{minted}

Con todo esto en orden se puede proceder a la compilación con Maven en una
simple linea.

\begin{minted}{bash}
 \$ mvn clean package
\end{minted}

Esto va a dar como resultado un nuevo directorio que contendrá las partes
resultantes de la compilación, en la raíz del directorio (con la configuración
que se tiene en el repositorio mencionado), se va a tener un archivo {\tt
.jar}, este archivo es el que se utilizará para ejecutar los trabajos en el
\gls{cluster}

\subsection*{Ejecución del paquete y resultados}

Una vez se finalizan los pasos anteriores, solamente queda ejecutar el paquete
obtenido sobre la plataforma Hadoop. Para continuar con esto, se asume la existencia de los
archivos que van a ser el input del Mapper en el directorio {\tt ./pylp/input}.
Es necesario recalcar que el directorio que funcionará como output no se
encuentre creado. Con todo esto en su lugar, se procede a ejecutar lo
implementado de la siguiente manera

\begin{minted}{bash}
\$ hadoop jar \
wc.jar \  # Archivo jar con la implementacion de Mapper y Reducer
WordCount  \ # Clase envoltorio dentro del proyecto
pylp/input/ \ # Directorio que tiene los inputs
pylp/output/ # Directorio que funcionará como output
\end{minted}

