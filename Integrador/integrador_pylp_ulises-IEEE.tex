\documentclass{IEEEcsmag}
\usepackage[colorlinks,urlcolor=blue,linkcolor=blue,citecolor=blue]{hyperref}
\usepackage[spanish]{babel}
\usepackage{upmath}
\usepackage[
  sorting=none,
  backend=biber,
  style=ieee,
]{biblatex}
\addbibresource{./bibliografia.bib}

% Glosario
\usepackage[acronym]{glossaries}
\makeglossaries

% Bordes
\usepackage{tcolorbox}

% \jvol{XX}
% \jnum{XX}
% \paper{8}
\jmonth{Julio}
\jname{Paradigmas y Leguajes de Programación}
\pubyear{2021}
% \newtheorem{theorem}{Theorem}
% \newtheorem{lemma}{Lemma}

\setcounter{secnumdepth}{0}
\begin{document}
\title{
  Apache Hadoop: Una guía paso a paso
 \newline
  \huge{{\it Instalación en un nodo simple
}}
}
\author{\input{info/nombre_autor.tex}}
      \affil{FCEQyN\\
      \input{info/institucion.tex}\\
      \input{info/correo.tex}
  }

\renewcommand{\abstractname}{Resumen} % Cambiamos el titulo, en vez de
\begin{abstract}
  Este documento es una continuación de lo charlado en el documento introductorio
al modelo \acrlong{mr} (\acrshort{mr})~\cite{ramirez2021}, aquí se procede a la
descripción de cómo configurar un cluster Hadoop para el procesamiento paralelo
y distribuido. En esta instancia la configuración se hará sobre un único nodo.

Algo imporntante a tener en cuenta es que los pasos seguidos para esta configuración
pueden quedar obsoletos ante cambios en distintos paquetes de los cuales
depende el presente tutorial. Se recomienda que ante la incapacidad de poder
realizar una tarea se revisen los canales de distribución oficiales para el
paquete en cuestión.


  \newline\newline
  \textit{\textbf{Palabras clave:} Multiproceso, programación concurrente,
  Procesamiento distribuido, Big Data, Apache Hadoop.
}
\end{abstract}

\maketitle

\chapterinitial{Parte I} % Inicia secciones en el doc
                              % con un titulo más grande
%--- Cuerpo -------
\section{Introducción} \label{sec:introduccion}
% Se dice que se ganó interés en los datos y el volumen de generación de los
% mismos, como así tambien los desafíos que proponen.
El manejo de gran cantidad de datos ha ganado atención en los últimos
años\cite{newvantage2019},
esto surge a raíz del ritmo acelerado que se generan estos datos y los 
diferentes escenarios que surgen a la hora de asegurar\cites{rawat2019,
kantarcioglu2019}, procesar (lo que se discute en este documento, aunque este a
su vez se basa en muchos que van ser citados) y almacenar\cites{padgavankar2014, agrawal2016, hongming2017}
los mismos.


% Se relaciona la cantidad de personas en el mundo y sus actividades con el
% creciente volumen de generación de datos
En la actualidad existen mas de 7.8 mil millones de habitantes en el planeta\cite{world2021}, 
un porcentaje considerable de esos habitantes son usuarios de internet como
describe Meeker\cite{meeker2019}. Aunque es cierto que la generación de datos viene 
tomando una tendencia creciente desde hace ya años \cite{lyman2003},
parte de este crecimiento abrupto de la última decada se puede atribuir, en parte,
al gran papel que la tecnología
fué adquiriendo con el pasar de los años y su constante evolución: comercio electrónico, publicidad
electrónica, pagos electrónicos, etc.\cite{meeker2019,pellicer2019},
las personas conectadas con varios dispositivos (\gls{wearables}, dispositivos \acrshort{iot},
celulares «especial énfasis», televisores, etc.), sensores produciendo una
ráfaga continua de datos, el avance y la gran disponibilidad de las 
redes móviles de cara a las redes de siguiente generación
(5g)\cite{sultan2018}, dan como resultado una enorme cantidad de datos.

% Se introduce el término "Big Data"
Esta enorme cantidad de datos acuño un nombre con el pasar de los años, el
término es \textit{big data}.

Shönberger y Culkier\cite{shonberger2016} brindan una síntesis concreta de lo
que es big data, sin embargo, como los mismos autores lo mencionan,
esto no es nada mas que el inicio.

\begin{displayquote}
    ``Datos masivos, o cosas que se pueden hacer a gran
    escala, pero no a una escala inferior, con el fin de extraer nuevas
    percepciones o crear formas de valor de tal forma que se transformen los
    mercados.''
\end{displayquote}

Transformar estos datos en valor viene ya llevándose a cabo hace décadas,
caracterizado con términos que fueron cambiando con el tiempo, conmunmente denominado
\textit{inteligencia de negocios} (\acrshort{bi}) sobre el cual hay extensa literatura y
gran variedad de escenarios en los que se aplicó el mismo con diferentes
enfoques\cites{britos2008, qazi2014, su2021, khozium2020, vaclav2021, phillips-wren2021, schwade2021}.
El big data, toma estos principios de la inteligencia de negocios y los aplica a una escala mayor.

Aquí, se explora el procesamiento del big data, los desafíos
en cuanto a procesamiento que surgieron y siguen surgiendo (aunque
también se mencionan brevemente algunas cuestiones relacionadas con los
desafíos que intervienten con la seguridad y el almacenamiento de la
información), los enfoques a la hora de tratar de encontrar valor a una
cantidad de datos masivo teniendo en cuenta diferentes atributos con los que
cuentan. 

% \subsection{Problemática}
% \label{sec:problematica}

En la sección \ref{sec:big_data} se describe en mayor profundidad lo que
significa el big data y cuales son sus características predominantes, es decir,
cuando alguien considera algo como big data, en la sección \ref{sec:procesamiento_paralelo} se
introduce lo que es el procesamiento paralelo, se hace especial énfasis
en cómo éste permite el tratamiento de los datos haciendo uso eficiente
de todos los recursos. A partir de ahí, en la sección 
\ref{sec:procesamiento_distribuido} se aplica esta idea de utilización
eficiente de los recursos y se los traslada a
un entorno mayor no limitado a una máquina y con un nivel de acoplamiento
inferior.

 % el bloque con csquotes mata la compilacion
\section{Big Data}
\label{sec:big_data}

Definir \textit{¿Qué es el big data?} es un trabajo no trivial, anteriormente, se brindó
un concepto que pretendía dar al lector un primer acercamiento al tema, sin
embargo este carece de completitud y especificidad. En
la literatura se demuestra que términos difusos como `grande' hacen aun más
difícil el determinar qué es y qué no es big data, esto supone tener que
definir variables que se van a tomar en cuenta en diferentes disciplinas con el
fin de poder determinar `¿Qué tan grande tiene que ser algo para ser grande?'.

La caracterización principal del big data y por lo tanto la respuesta a la
pregunta anterior se tiene en lo que la literatura llama \textit{las 3
Vs}\footnote{Existen autores que consideran que son
4\textit{V}s\cite{wibowo2019, ghasemaghaei2019}, 5\textit{V}s
\cites{hitzler2013, yin2015}, e incluso autores que
consideran más \textit{V}s~\cite{bonner2017}. Sin embargo, cabe
destacar que estas dos opiniones incluyen las 3\textit{V}s mencionadas
anteriormente.}\cites{bd_ibm, bd_aws, bd_oracle}. Antes de continuar con las \textit{V}s, es conveniente realizar una
aclaración importante: En algunos casos la definición o caracterización aceptada del big data
con las \textit{V}s \textit{no es aplicable}, por ejemplo, en el ámbito de la medicina, Baro {\it et. al}
\cite{baro2015} determinan que la cantidad de datos necesarios para ser grande
se satisface si se cumple que,


\begin{equation} \label{eq-baro2015-1}%
  \log_{10}(n * p) \geq 7 \\
\end{equation}

En donde se tiene que,

\begin{tabular}{l l}
$n$ & Cantidad de \textit{individuos estadísticos}\\
$p$ & Cantidad de variables a ser analizadas \\
    & dentro del \gls{dataset}\\
\end{tabular}

% Siendo $n$ la cantidad de \textit{individuos estadísticos}
% y $p$ la cantidad de variables a ser analizadas dentro del \gls{dataset}
% compuesto por un conjunto de artículos.


En este caso específico analiza el significado de ¿Qué es ser big data? para un
área en particular, \textit{la medicina}\footnote{cabe destacar que no es posible asumir que este
análisis llevado a cabo por los autores es trasladable a todas las
disciplinas}, aquí se busca enfocar y delimitar el concepto de
grande al área de la salud. El motivo principal detrás de esto es que luego de
estudiar los diferentes casos (entendiendo como \textit{casos} las
publicaciones en revistas científicas orientadas exclusivamente a la medicina)%a la hora de trabajar con los datos dentro de
en un ambiente orientado a la salud, no se satisfacen los criterios de
masividad\footnote{Comparado con la cantidad de consultas que recibe Google, la
cantidad de reseñas escritas en Amazon, la cantidad de comentarios que se
publican en Twitter, etc.}, pero aún así existe en el corpus material acerca
del big data asociado a la medicina, por lo que se hace útil poseer una
definición mas ajustada a dicha disciplina.


\subsection{Volúmen}
\label{ssec:volumen}

El volúmen hace referencia a la cantidad de datos dentro de un \gls{dataset}\cites{ghasemaghaei2019,
ghasemaghaei2021}, considerado por algunos autores como la característica que
brinda los mayores desafíos para el tratamiento con el big data\cite{che2013}.

Se habla de cantidad, y nuevamente se tiene esta idea de \textit{¿Qué tantos datos hacen
falta para tener un volúmen lo suficientemente masivo como para ser considerado
grande?}

Hacia el 2010, el mundo había creado 1\acrshort{zb}\footnote{NOTA: 1\acrshort{zb} equivale a $1 \times
10^{9}$ terabytes, es decir 1000000000\acrshort{tb}.} de datos, se menciona
esto para tener
una idea aproximada de la cantidad de datos manejados hace DIEZ AÑOS ATRAS! (una
eternidad en el mundo de las tecnologías de la
información),  y se estimaba que para el 2020 el mundo habría creado 40\acrshort{zb}\cite{lam2017}, 
sin embargo, para 2018 ya se tenían 33\acrshort{zb}, lo cual llevó al IDC a 
realizar nuevas consideraciones y se terminó estimando
que para 2025 se tendrían 175\acrshort{zb}\cite{forbes2020}. Es decir,
que en el tramo de 15 años, lo que se consideraba inconmensurable al inicio
multiplicaría su tamaño por 175. Hablar de estas magnitudes se
va naturalizando a medida que pasa el tiempo, ya en el 2013 se iba haciendo
cotidiano hablar de Petabytes (\acrshort{pb}) de cara al uso de
Exabytes (\acrshort{eb})\cite{che2013}. 


\subsection{Variedad}
\label{ssec:variedad}

Esta es otra de las características principales dentro del big data,
y hace referencia a la variedad de datos que conforman un \gls{dataset}. Este
fenómeno tiene lugar por el simple hecho de que existen casi ilimitadas fuentes
que pueden contribuir a un \gls{dataset}, el cual puede estar compuesto por
diferentes tipos y formas de representaciones\cite{che2013} lo cual hace al big
data grande y estos pueden ser resumidos en estructurados, semi-estructurados y
no-estructurados\cite{sagiroglu2013}.

Los datos estructurados y los semi estructurados caben en los Sistemas de Administración de Bases de
Datos (\acrshort{dbms}) y/o Data-Warehouses (\acrshort{dw}), acá se mencionan
los semi estructurados porque estos estan compuestos por archivos tales como
XML, en donde se tienen etiquetas para separar diferentes elementos de datos.
En cuanto a los datos no estructurados se puede mencionar a la web como una
fuente importante de los mismos, un ejemplo primordial aquí es el denominado
\gls{clickstream}\cites{russom2011, albert2010-c05}, aunque menciones
honorables van además para los mensajes de textos que se obtienen de las
compañias de celulares, datos generados por sensores en dispositivos
\acrshort{iot}, etc.

\subsection{Velocidad} \label{ssec:velocidad}

En la introducción se hizo mención de la utilidad de herramientas para
análisis de datos, más específicamente, se habló de Almacenes de Datos o 
\acrlong{dw}/\acrlong{edw} (\acrshort{dw}/\acrshort{edw}), y su contribución
superlativa a la Inteligencia de Negocios (\acrshort{bi}) mencionada en la
Sección \ref{sec:introduccion}.

Este sistema utilizado en \acrshort{bi} es una respuesta para múltiples
desafíos que se tienen dentro de una organización\cite{datalytics2}, en este
documento se hace énfasis concretamente en uno de los motivos, el 
\textit{para dar soporte a las desiciones que se toman a nivel estratégico}.
Para lograr ese soporte, la organización 
sigue una metodología que le va a permitir aplicar estas herramientas
analiticas con el fín de extraer el valor de los datos. La gente de
Datalytics\footnote{https://www.datalytics.com/}, menciona a lo largo de varias
charlas que componen el ciclo de ``DataSchool'', la importancia de estas
arquitecturas clásicas, sin embargo, el acercamiento clásico a la creciente
cantidad de datos brinda más desafíos, particularmente el que más resuena en el
contexto de este documento es el hecho de que esta arquitectura tradicional es 
\textit{lenta para reaccionar}\cite{datalytics3}.

El concepto de velocidad es esto, y es tan importante como los que se hablaron
anteriormente. La velocidad a la que se pueda realizar un análisis concreto y
tomar decisiones basadas en los resultados es clave. El análisis de un
fragmento de información acerca de la competencia, datos estadísticos realizados
por algún organismo ajeno a la organización que tiene una periodicidad anual
(ej: \acrshort{indec}), etc. posee una gran potencialidad, sin embargo, si no
se analiza oportunamente, el valor que pueda existir en esa información se
pierde.




% Alguna conclusión probablemente para el tema de BD

\section{Procesamiento Paralelo}
\label{sec:procesamiento_paralelo}
%Pequeña introduccion de que es el procesamiento paralelo
% - Explicar que existen tres caminos de paralelizacion, aunque la tercera
%   la que está basada en redes de neuronas, es un sistema que difiere al que
%   se viene viendo.

Diversos autores\cite{moldovan93, trobec2018} describen en profundidad diferentes modelos que
extienden a la arquitectura tradicional de Von Neumann para lograr
paralelización\footnote{No es el objetivo de este documento dar una descripción minuciosa de los
diferentes modelos teóricos de paralelización.}. Esta se logra mediante la separación de una tarea $T$, en subtareas
$T_{1}, T_{2}, \ldots , T_{n}$ (Diferentes autores adoptan este concepto con el
término de \textit{\gls{pipelining}}~\cite{hyde98, trobec2018}). Luego para lograr el cometido, 
más de una de las $T_{n}$ tareas debe realizarse al mismo tiempo, sin malinterpretar la
realización de una tarea de manera muy velóz con realizarla al mismo tiempo que
otras~\cite{hyde98}.

Las diferentes soluciones para procesamiento paralelo se desenvolvieron hasta
llegar a presentar uno de los siguientes tres tipos de
paralelismo\cite{trobec2018}:

\begin{itemize}
\item Los {\bf Sistemas de Memoria Compartida} que se componen de múltiples unidades de procesamiento
unidas a una memoria.
\item Los {\bf Sistemas Distribuidos} que se componen de equipos con sus propias unidades de procesamiento
y memoria, comunicados a traves de una conexión de red de alta velocidad. Este
es el caso que nos compete en este documento y es tratado en la Sección
\ref{sec:procesamiento_distribuido}.
\item Las {\bf Unidades de Procesamiento Gráfico} utilizadas como co-procesadores por su
capacidad de paralelizar grandes cantidades de operaciones.
\end{itemize}

Diferentes enfoques existen a la hora de aplicar el \gls{pipelining} mencionado
anteriormente, y estos no son recientes sino que datan desde la década de 1990.
Con el pasar del tiempo se fueron creando más enfoques y especializando los
casos generales que se tenían en un principio. Ejemplo de algunas
especializaciones son los enfoques que se ven en la cátedra de Paradigmas y Lenguajes 
de Programación\cite{pylp1}, en donde se denomina a este problema
\textit{Descomposicion}, y de ahí surgen las diferentes especializaciones:
Dominio, Funcion, Recursiva, Mixta.

Para este documento solamente se tienen en cuenta dos de las 3 que se 
proponen por Fountain~\cite{fountain1994}: Paralelización de Datos y
Paralelización de Funciones.

La paralelización de datos y la paralelización de funciones puede encontrarse 
en diferentes niveles de la arquitectura
del computador, algunos ejemplos van desde, el procesador como lo describe
Hyde~\cite{fountain1994}, pasando por el compilador~\cite{clark1997}, llegando
a la aplicación~\cite{wolters1995}, etc. Este 
documento cubre únicamente a casos que son aplicados a nivel de aplicación.

\subsection{Paralelización de Datos}
\label{ssec:paralelizacion_de_datos}

\begin{tcolorbox}
La idea principal detrás de este enfoque a la hora de aplicar el {\it pipelining}
consiste en que un programa secuencial puede ser transformado a un
programa paralelo, realizando ejecuciones de copias idénticas de dicho programa como
tareas separadas, a las cuales se les brinda simplemente parte de los datos
iniciales~\cite{haveraaen2000}.
\end{tcolorbox}

En la figura \ref{fig:sin_paralelizacion_datos} se presenta una aplicación que
está corriendo sobre un núcleo aleatorio en un ordenador. En este ejemplo no se tiene
presente la paralelización, $D$ representa los datos sobre los cuales está
trabajando la \textit{Aplicación}.

\begin{figure}[h]
  \centering
  \includegraphics[width=0.3\linewidth]{figuras/procesamiento_paralelo_sin_paralelizacion_datos.png}
  \caption{Aplicación sin paralelización}
  \label{fig:sin_paralelizacion_datos}
\end{figure}

En la figura \ref{fig:con_paralelizacion_datos} se presenta la misma
aplicación (esta vez se la denomina \textit{app}). En este ejemplo
los datos representados por $D$ anteriomente, se
dividien en $D_{n}$ secciones más pequeñas que son distribuidas a distintos
núcleos de procesamiento \footnote{en el ejemplo se utilizan 4, aunque pueden ser más
núcleos dentro de un procesador}. Una copia idéntica de la aplicación (Figura
\ref{fig:sin_paralelizacion_datos}) se encuentra corriendo en cada uno de los
cuatro núcleos del ordenador, cada una de estas copias idénticas se
encuentra trabajando con datos distintos (ya que a cada una de las copias le
toca una sección $D_{n}$ diferente).

\begin{figure}[h]
  \centering
  \includegraphics[width=0.3\linewidth]{figuras/procesamiento_paralelo_con_paralelizacion_datos.png}
  \caption{Aplicación con paralelización de datos}
  \label{fig:con_paralelizacion_datos}
\end{figure}

Una interpretación alternativa se puede ver en la figura
\ref{fig:con_paralelizacion_datos_alt}, este gráfíco se puede leer como ``Una
aplicación que se encuentra corriendo sobre distintos nucleos, con distintos
fragmentos de los datos originales''

\begin{figure}[h]
  \centering
  \includegraphics[width=0.3\linewidth]{figuras/procesamiento_paralelo_con_paralelizacion_datos_alt.png}
  \caption{Aplicación con paralelización de datos - Representación alternativa}
  \label{fig:con_paralelizacion_datos_alt}
\end{figure}


En la bibliografía existen implementaciones en diferentes sub áreas de las Ciencias de la
Computación con este enfoque presente, el procesamiento de
imágenes~\cite{pang2009}, el manejo de bases de
datos~\cite{zakharov2019} y la inteligencia artificial~\cite{hajj2015} por
nombrar algunos ejemplos. 

Para un ejemplo simple y concreto se propone lo siguiente: 

\begin{tcolorbox} \label{ej:1} 
  Se presenta la tarea de encontrar el valor mínimo dentro de un arreglo 
  $A = [a_{0}, a_{1}, \ldots , a_{n-2}, a_{n-1}]$.
\end{tcolorbox}

Se puede proceder sin aplicar paralelización de datos, y realizar una
aplicación que recorra los $n$ elementos del arreglo, realice las
comparaciones necesarias y devuelva el valor mínimo, o se puede
optar por la opción con paralelización de datos, una solución con este enfoque 
será fraccionar el arreglo con $A$ en, por ejemplo, 2 partes $A_{1}$ y
$A_{2}$. Luego se procede a distribuir estas dos partes $A_{1}$ y
$A_{2}$ a diferentes instancias de la aplicación que calcula el mínimo, 
de esta manera se tendrán dos núcleos de procesamiento que trabajarán en un
problema de menor tamaño (ya que
cada uno va a trabajar únicamente con el 50\% de los datos), y al final
solamente hará falta realizar una comparación entre los mínimos que
encuentren las dos instancias de la aplicación.

Cabe destacar que algunos problemas pueden surgir a la hora de particionar
datos si esto se hace sin nignun análisis, un ejemplo de esto puede ser
considerado en el escenario en donde ocurre
la división de una imagen en partes más pequeñas para su análisis en forma
paralela~\cite{oshitani1999}. También se hace presente la limitación que impone
la ley de Amdhal\footnote{``El incremento máximo de velocidad (la cantidad máxima de
procesadores que pueden ser utilizados de manera efectiva) es la inversa de la
fracción de tiempo que la tarea toma para finalizar en un solo
hilo'' \cite{rodgers1985}}.
% \begin{figure}[h]
  \centering
  \label{fig:paralelizacion_datos}
  \resizebox{!}{5cm}{%
    \begin{tikzpicture}
      % Dato
      \draw (0,4.75) rectangle (5,8.75) node[midway,blue]{\texttt{Datos}: $1,2,3,4,5,6,7,8,9,10$};

      % Datos
      \draw (9, 10) rectangle(10.5,11.5) node[midway,blue]{$1,2$};
      \draw (9, 8) rectangle (10.5,9.5) node[midway,blue]{$3,4$};
      \draw (9, 6) rectangle (10.5,7.5) node[midway,blue]{$5,6$};
      \draw (9, 4) rectangle (10.5,5.5) node[midway,blue]{$7, 8$};
      \draw (9, 2) rectangle (10.5,3.5) node[midway,blue]{$9, 10$};
    \end{tikzpicture}
  }
  \caption{Concepto paralelización de datos}
\end{figure}


\subsection{Paralelización de Funciones}
\label{ssec:paralelizacion_de_funciones}

\begin{tcolorbox}
  Este enfoque a la paralelización toma otro camino a la hora de aplicar el
  \textit{pipelining}, este acercamiento
  no fracciona los datos sino que transforma la tarea a un gráfo de 
  dependencias en donde cada nodo corresponde a una operación a ser realizada
  para cumplir la tarea. Y lo que trata de hacer es ejecutar la mayor
  cantidad de tareas al mismo tiempo~\cites{meng2013, liu2019, zhou2020,
  lin2021}.
\end{tcolorbox}

Esta idea de distribuir los datos a diferentes unidades funcionales que se
encarguen de realizar una tarea sobre los datos mencionados se presenta en la
figura \ref{fig:con_paralelizacion_funciones}.

% En la figura \ref{fig:con_paralelizacion_datos} se presenta la misma
% aplicación (esta vez se la denomina \textit{app}). En este ejemplo
% los datos representados por $D$ anteriomente, se
% dividien en $D_{n}$ secciones más pequeñas que son distribuidas a distintos
% núcleos de procesamiento \footnote{en el ejemplo se utilizan 4, aunque pueden ser más
% núcleos dentro de un procesador}. Una copia idéntica de la aplicación (Figura
% \ref{fig:sin_paralelizacion_datos}) se encuentra corriendo en cada uno de los
% cuatro núcleos del ordenador, cada una de estas copias idénticas se
% encuentra trabajando con datos distintos (ya que a cada una de las copias le
% toca una sección $D_{n}$ diferente).

Aquí se puede observar que lo que se subdivide no son los datos, sino que lo que
se está subdividiendo es la funcionalidad que compone a la tarea $T$, aquí cada
$o_{n}$ es una operación claramente delimitada dentro de la aplicación, y $D$
representa a los datos con los que se va a trabajar.

\begin{figure}[h]
  \centering
  \includegraphics[width=0.65\linewidth]{figuras/procesamiento_paralelo_con_paralelizacion_func.png}
  \caption{Aplicación con paralelización de funciones}
  \label{fig:con_paralelizacion_funciones}
\end{figure}


Un ejemplo concreto para este enfoque se presenta a continuación:

\begin{tcolorbox}
  Dada una imagen, se presenta la tarea de determinar si en dicha imagen se
  encuentra presente algún rostro, vehículo, gato o planta.
\end{tcolorbox}

Por lo que se puede decir que la tarea $T$ se compone de la siguiente manera

\begin{equation} \label{eq-tareas-nodos}%
  T = [o_{0}, o_{1}, o_{2}, o_{3}]
\end{equation}

En donde se tiene que,

\begin{tabular}{l   l}
$o_{0}$ & Buscar rostro \\
$o_{1}$ & Buscar vehículo \\
$o_{2}$ & Buscar planta \\
$o_{3}$ & Buscar gato\\
\end{tabular}

\vspace{1cm}

Cada $o_{n}$ en la ecuación \ref{eq-tareas-nodos} representa un nodo del gráfo
de mencionado anteriormente. Ahora toca analizar y determinar
cuales de las $o_{n}$ operaciones se pueden realizar al mismo tiempo.

Una cuestion esencial en este ejemplo es identificar el hecho de que ninguna
operación $o_{n}$ depende de alguna operación.
% $o_{m}$ $\forall m, n \in
% \mathbb{N}| 0 \leq m, n \leq 3$
Esto es visible a la hora de realizar diversas preguntas con
respecto al contexto, 
\begin{itemize}
    \item ¿Es necesario determinar si existe o no un rostro para poder iniciar
      el análisis y determinar si la imagen contiene un vehiculo, gato o
      planta? No, entonces, {\it Buscar rostro} es independiente a las demás
      operaciones.
    \item ¿Es necesario determinar si existe un vehiculo en la imagen para
      poder iniciar el análisis y determinar si la imagen contiene un rostro,
      gato o planta? No, entonces, {\it Buscar vehículo} es independiente a las
      demás operaciones.
\end{itemize}

El análisis anterior debe continuar hasta que se analicen las $o_{n}$
operaciones. Habrán veces en donde dos o más operaciones no podrán ser ejecutadas de
manera paralela, esto puede darse porque una (también pueden ser varias) de ellas 
depende de los resultados que se obtengan de otra u otras operaciones.

En el caso del ejemplo para la sección, esta apreciación y el análisis que se
realizó permite definir que todas las operaciones que componen a la tarea $T$
se pueden ejecutar de manera paralela por el mismo dato.

\section{Procesamiento Distribuido}
\label{sec:procesamiento_distribuido}

Este tiene muchas similitudes con los sistemas de Procesamiento Paralelo, es
más, es válido decir que los Sistemas Paralelos y los Sistemas Distribuidos son
un entrelazamiento de redes, que tienen la caracteristica de ser
continuas\footnote{
o un \gls{continuum} de redes.
\url{https://dictionary.cambridge.org/es/diccionario/ingles/continuum}}, lo que
significa que el parámetro que determina si el sistema es distribuido es el promedio de
distancias entre los nodos de procesamiento~\cite{hyde98}. El desafío que
propone Hyde, y también diferencia a los dos sistemas es el de poder determinar el
estado exacto de todo el sistema. Este es un desafío no-trivial, ya que dicho
estado no puede ser computado
instantaneamente por el hecho de que cada operación dentro de la red podría
requerir travesías por la misma.

Esta caracteristica mencionada hace que sea muy complicado establecer una unica
definición de que significa un sistema distribuido. Para el propósito del
documento se hace útil una definición brindada por 
Tanenbaum \textit{et. al.}~\cite{tanenbaum2007a}, la cual sugiere que,

\begin{displayquote}
  ``Un sistema distribuido es una colección de computadoras independientes que
  parecen ser un único sistema coherente ante los usuarios.''
\end{displayquote}

Aquí hay dos cuestiones extremadamente importantes que son de interés para el
documento. La primera, se habla de una \textit{colección de computadoras} y se
hace énfasis en el hecho de que estos ordenadores son independientes, lo cual
llevan a otro aspecto a considerar para esta colección de máquinas, el hecho de
que son \textit{fácilmente escalables} y \textit{tolerante a fallos}. La
segunda, se habla de los usuarios ya sea que se esté hablando de \textit{personas
o programas} y como para estos es transparente el hecho de que están lidiando con
un sistma distribuido, es decir, estos tienen la sensación de utilizar un
sistema compuesto por un único ordenador.

Además de estas dos cuestiones que se hacen presentes en la definición
presentada, existen otros aspectos dentro de los sistemas distribuidos que
son importantes para el estudio de los mismos en profundidad, sin embargo, no es el
objetivo de este documento estudiar los sistemas distribuidos en profundidad,
a continuación se presenta un punto de referencia para que el
lector pueda seguir estudiando a los sistemas distribuidos y sus diferentes
aspectos, características y desafíos los cuales no son
pocos~\cite{tanenbaum1994, tanenbaum2007, siberschatz2013}.

\printbibliography[keyword={recomendacion_sistemas_distribuidos}, title={Referencias para 
el estudio de Sistemas Distribuidos}]

\section{MapReduce}
\label{ssec:mapreduce}

En las secciones anteriores, se habló de Procesamiento Paralelo y Distribuido,
se mencionaron algunas características (particularmente las más afínes a este
documento), en esta sección, se busca aplicar las cuestiones que se mencionaron
con anterioridad, y empezamos a hablar del núcleo del documento
\textit{MapReduce}.

MapReduce (\acrshort{mr}), es un modelo de programación; una forma de hacer las cosas; un
paradigma. Este modelo fue introducido por Google en la primer mitad del
2000~\cite{dean2004} y la implementación del mismo se asocia con grandes 
\glspl{dataset} y es
aplicable a una gran cantidad de tareas del mundo real~\cite{dean2008}. En esta
sección se avanza sobre una introducción acerca del modelo que presenta
MapReduce y 
luego se aplicarán los conceptos en una de las varias herramientas que aplican
el modelo en cuestion, la herramienta a la que se hacer referencia es un 
\Gls{framework} para Big Data, su nombre es 
\textit{Hadoop}\includegraphics[height=10pt]{figuras/logos/hadoop_logo.png}.

\subsection{Funciones \texttt{Map} y \texttt{Reduce}}

\paragraph{Un vistazo general:}
en este modelo los usuarios que implementan \acrshort{mr} definen dos
funciones principales, una función \textit{Map} y una función \textit{Reduce},
el sistema que da soporte a la aplicación implementada se encarga de
paralelizar la ejecución de la misma a lo largo y ancho de un sistema
distribuido (\gls{cluster}) de ordenadores, este \gls{cluster} cuenta con las
características de los sistemas distribuidos mencionados en la sección
\ref{sec:procesamiento_distribuido}. Sistemas como estos permiten que Google
(en el 2008) pueda procesar 20\acrshort{pb} por día~\cite{dean2008}.

Las implementaciones de diferentes cargas de trabajo llevadas a cabo en el 
2004 por Google con este nuevo modelo de programación sentaba sus bases sobre un
desarrollo anterior extremadamente importante, el Sistema de Archivos de
Google (\acrshort{gfs}, por sus siglas en inglés)~\cite{ghemawat2003}, el cual,
en resumidas cuentas, es un sistema de archivos distribuidos (lo cual está en
línea con lo presentado anteriormente con respecto a la ``paralelización de datos'').

\subsubsection{Map}
\label{sssec:map}

\paragraph{Resumen} Esta función {\tt map}, se distribuye automáticamente sobre el sistema
distribuido, y puede trabajar ya sobre {\it chunks} (partición) de datos
que se encuentran dentro de los diferentes nodos, a los cuales se les distribuye el
procesamiento de dicha función. 

\begin{figure}[h!]
  \centering
  \includegraphics[width=0.75\linewidth]{figuras/MapReduce_mapper_phase.png}
  \caption{MapReduce: Fase {\tt map}}
  \label{fig:mapreduce_mapper_phase}
\end{figure}



\begin{enumerate}
\item {\bf fork:} El sistema distribuido se encarga de realizar la replicación
      de la aplicación del usuario sobre el conjunto de máquinas, esta
      aplicación se divide en copias que son {\it workers} y una especial que
      se denomina {\it master}\footnote{Esto es análogo a lo que se estudia en
      la cátedra con OpenMPI.}. Este último es el encargado de dividir las
      tareas a las copias que son workers.
\item {\bf Asig. Map:} El {\it master} asigna las tareas a los workers. Los
      cuales, cabe aclarar, trabajan sobre un o varios chunks de datos locales al
      nodo.
\item {\bf Lectura:} Un {\it worker} al que se le asigna una tarea {\tt map},
      lee los contenidos de los {\it archivos de entrada}, interpreta estos
      datos en formato clave-valor y se encarga de distribuir estos pares a la
      función {map}. Los valores que se produzcan por la función {\it map}, 
      se almacenan en un buffer de memoria.
\item {\bf Escritura Local:} Periodicamente, los pares que se almacenan en el
      buffer de memoria, pasan a ser escritos al disco de manera particionada. Las 
      ubicaciones de las particiones se comunican al {\it master} el cual será
      responsable más adelante de distribuir esta info. a los {\it workers} que
      son parte del {\it reduce}.
\end{enumerate}

\subsubsection{Reduce}
\label{sssec:reduce}

\paragraph{Overview} Esta función, al igual que los {\it mappers} se distribuye
automáticamente a través del sistema distribuido y trabaja sobre los archivos
intermedios que resultan de los mencionados {\it mappers}, los {\it workers}
que se encargan de realizar la etapa {reduce} leen estos archivos desde los
nodos de los {\it mappers} y se encargan de procesar los datos asociados a cada
clave-valor en orden.

Las etapas que son llevadas a cabo en la fase de {\it reduce}, se pueden ver en
la figura \ref{fig:mapreduce_reducer_phase}.


\begin{enumerate}
\setcounter{enumi}{1}
\item {\bf Asig. Reduce:} El {\it master} se encarga de distribuir la carga de
      reducer información correspondiente a la ubicación de los archivos 
      intermedios a los {\it workers} que trabajan como reducers.
\setcounter{enumi}{4}
\item {\bf Lectura Remota:} Al recibir la información del {\it master} acerca
      de la ubicación de los {\it archivos intermedios}, los {\it workers}
      proceden a la lectura de los mismos a través de RPC. El worker ahora
      procede a realizar las tareas de {\it reduce}.
\item {\bf Escritura:} Luego de haber procesado los datos de las claves que le
      fueron asignadas al {\it reducer}, este pasa a guardar los resultados a
      los archivos de salida {\it S1} y {\it S2}.
\end{enumerate}

\begin{figure}[h]
  \centering
  \includegraphics[width=0.75\linewidth]{figuras/MapReduce_reducer_phase.png}
  \caption{MapReduce: Fase {\tt reduce}}
  \label{fig:mapreduce_reducer_phase}
\end{figure}




\section{Hadoop}
\label{sec:hadoop}

\subsection*{Introducción}
Apache Hadoop es una herramienta que surge luego de dos grandes contribuciones
de las que se estuvieron hablando en las secciones anteriores, estas dos son,
el modelo de programación MapReduce~\cite{dean2004, dean2008}, y
\acrshort{gfs}~\cite{ghemawat2003}.

Este es un \gls{framework} para Big Data implementado en Java, esta
implementación consiste de dos capas principales, la primer capa es para
almacenamiento de datos, esta capa es
 un sistema de archivos distribuidos que está basado en \acrshort{gfs}, 
que se denomina \acrlong{hdfs} (\acrshort{hdfs}), la segunda capa, ataca el
problema del procesamiento de los datos y se denomina {\it Hadoop MapReduce
\Gls{framework}}~\cite{lee2012}.

Este \gls{framework} trabaja exclusivamente sobre pares clave valor ({\tt «K,
V»}), se puede resumir este funcionamiento de la siguiente manera~\cite{}:

\begin{tcolorbox}
  \centering
  {\tt input} <k1, k2>$\rightarrow$ {\tt mapper} $\rightarrow$ <k2, v2> $\rightarrow$
  {\tt reducer} $\rightarrow$ <k3, v3>
\end{tcolorbox}

\subsection*{Implementación: Mapper y Reducer}

Para demostrar de manera menos abstracta y teórica el funcionamiento del
framework, ahora se procede a la implementación de un {\it mapper} y un {\it
reducer} extremadamente sencillos. El propósito del código brindado en este 
ejemplo es el de contar
la cantidad de palabras en diferentes archivos distribuidos en el
\acrshort{hdfs}. En los listings \ref{listing:hadoop_mapper} y
\ref{listing:hadoop_reducer} se tiene el código respectivo para cada uno de estos. 

La implementación completa (no solamente del mapper y el reducer) se puede encontrar
en el Anexo: {\bf Implementación sobre HDFS}.

\begin{listing}[ht]
\inputminted[
  frame=lines,
  bgcolor=codebg,
  framesep=2mm,
  baselinestretch=1.2,
  fontsize=\footnotesize,
  linenos]{java}{cuerpo/codigo/TokenizerMapper.java}
\caption{Hadoop MapReduce: Mapper}
\label{listing:hadoop_mapper}
\end{listing}



Lo primero a notar con la implementación de la clase creada,
es el hecho de que extiende de {\tt Mapper}~\cite{apache_hadoop_mapper}. 

\mint{java}|extends Mapper<Object, Text, Text, IntWritable>|

La clase Mapper abstrae dentro del \gls{framework} la distribución de tareas a
los diferentes nodos en la etapa {\it Map}, esta clase se instancia una vez por
cada {\it InputSplit}\footnote{Representa el/los datos a ser procesados por un mapper
individual} \footnote{https://hadoop.apache.org/docs/r2.7.4/api/org/apache/hadoop/mapreduce/InputSplit.html}.

Lo siguiente a prestar atención, es la reimplementación del método {\tt map},
este método se aplica a cada registro dentro del InputSplit, la funcionalidad
por defecto es devolver el mismo registro.

\begin{minted}{java}
public void map(Object key, Text value, Context context) 
    throws IOException, InterruptedException {
  StringTokenizer itr = new StringTokenizer(value.toString());
  while (itr.hasMoreTokens()) {
    word.set(itr.nextToken());
    context.write(word, one);
  }
}
\end{minted}

Los primeros dos parametros reiteran lo mencionado anteriormente, el uso de clave-valor
{\tt (K, V)}. Aquí, {\tt Object key} representa a la clave del InputSplit
que le toca al mapper instanciado, y de la misma manera, {\tt Text value}
representa al dato del InputSplit asociado al mapper. Lo que queda pendiente es
el {\tt Context context}, este permite la comunicación con el resto del {\it sistema
distribuido}, este objeto contiene información referente a la configuarción del
trabajo actual, reporte de progreso, mensajes de estado a nivel de aplicación, 
\glspl{heartbeat}.

La información contenida en {\tt value} se transforma en un iterador, que
contiene en cada uno de sus elementos a una palabra (para este caso
particular), al recorrer este iterador, se escribe otro par {\tt (K2, V2)}, la
clave pasa a ser la palabra, y el contador pasa a ser un numero (1), en un
ejemplo concreto esto significa que, para el par {\tt (hadoop, 1)} la palabra
``hadoop'', tiene 1 repetición. Al terminar el recorrido se finaliza con la
``Emisión'' de estos pares (K2, V2) intermedios. Ahora la etapa {\it Reduce}
puede acceder a estos pares (K2, V2).

\begin{listing}[ht]
\inputminted[
  frame=lines,
  bgcolor=codebg,
  framesep=2mm,
  baselinestretch=1.2,
  fontsize=\footnotesize,
  linenos]{java}{cuerpo/codigo/IntSumReducer.java}
\caption{Hadoop MapReduce: Reducer}
\label{listing:hadoop_reducer}
\end{listing}



Al igual que en el mapper, acá se inicia prestando atención a la definición de
la clase {\it Reducer}, en este caso.

\mint{java}|extends Reducer<Text,IntWritable,Text,IntWritable> {|

Al igual que la etapa anterior, el reducer, acepta pares (K2,V2). La diferencia
es que estos pares, son el resultado de las operaciones realizadas por el 
{\it Mapper} en el listing \ref{listing:hadoop_mapper}.

\begin{minted}{java}
public void reduce(Text key, Iterable<IntWritable> values, Context context)
    throws IOException, InterruptedException {

  int sum = 0;
  for (IntWritable val : values) {
    sum += val.get();
  }
  result.set(sum);
  context.write(key, result);
}
\end{minted}

Nuevamente, aquí se cuenta con la clave {\tt Text key} y el valor\newline{\tt
Iterable<IntWritable> values}, este último, posee todas las instancias de una
clave específica. Para ilustrar este contenido de mejor manera, se presenta un
ejemplo, aquí se tiene el par,

\begin{tcolorbox}
\centering
{\tt (K, V) = (``hadoop'', Iterable [1, 1, 1, 1, 1])}
\end{tcolorbox}


Descomponiendo el ejemplo anterior se tiene lo siguiente, 

\begin{tcolorbox}
\centering
{\tt (``hadoop'', 1)} \\
{\tt (``hadoop'', 1)} \\
{\tt (``hadoop'', 1)} \\
{\tt (``hadoop'', 1)}
\end{tcolorbox}

Estos múltiples pares (K, V) resultaron de la etapa mapper (que se
recuerda que pueden estar en {\it archivos intermedios} separados en distintos
nodos), que se unen en la etapa número 5, explicada en la subsección \ref{sssec:reduce}
{\it Reduce}.

Resumiendo el significado puntual de lo visto en términos simples es,  

\begin{tcolorbox}
  Iterable<IntWritable> values = {\bf Todos los valores con la clave ``hadoop''}
\end{tcolorbox}

Recordando que los valores son siempre 1, estos son siempre el mismo valor
por el problema que se está atacando con el mapper en este ejemplo particular,
no necesariamente son {\bf siempre} 1 en todos los problemas.




\printbibliography
%--- Glosario ---
\clearpage
% Terminos
\newglossaryentry{wearables}{
  name=wearables,
  description={\textit{Vestible}, En el contexto de la tecnología, hace referencia a un
  dispositivo que se pueda \textit{vestir}. Ej: relojes inteligentes}
  }

\newglossaryentry{dataset}{
  name=dataset,
  plural=datasets,
  description={\textit{Conjunto de datos}, sobre los cuales se realizan experimentos.
  Las conclusiones que los investigadores definan sobre un cierto tema, se da
  mediante los experimentos realizados sobre uno o más conjuntos de datos}
  }

\newglossaryentry{clickstream}{
  name=clickstream,
  description={\textit{Flujo de clicks}, es una bitacora detallada de como los usuarios
  navegan una página web al realizar una tarea}
  }

\newglossaryentry{continuum}{
  name=continuum,
  description={\textit{Continuo}, algo que cambia gradualmente o en pequeños
  incrementos sin ningún pico evidente}
  }

\newglossaryentry{pipelining}{
  name=pipelining,
  description={En Ciencias de la Computación, este hace
  referencia a una organización en la cual pasos sucesivos de una secuencias de
  instrucciones son ejecutadas por diferentes módulos, esto para que otra
  instrucción pueda iniciar antes que una instrucción anterior finalice}
  }

\newglossaryentry{framework}{
  name=framework,
  plural=frameworks,
  description={\textit{Marco de trabajo}, es un conjunto estandarizado de 
  conceptos, prácticas y criterios para enfocar un tipo de problemática particular
  que sirve como referencia, para enfrentar y resolver nuevos problemas de índole 
  similar}
  }

% Siglas
\newacronym{iot}{IoT}{Internet of Things}
\newacronym{bi}{BI}{Business Intelligence}
\newacronym{dbms}{DBMS}{Data Base Management Systems}
\newacronym{dw}{DW}{Data Warehouse}
\newacronym{edw}{EDW}{Enterprise Data Warehouse}
\newacronym{indec}{INDEC}{Instituto Nacional De Estadística y Censos}
\newacronym{mr}{MR}{MapReduce}
\newacronym{tb}{TB}{Terabyte}
\newacronym{pb}{PB}{Petabyte}
\newacronym{pram}{PRAM}{Parallel Random-Acces Machine}
\newacronym{zb}{ZB}{Zettabyte}
\newacronym{eb}{EB}{Exabyte}


\chapterinitial{Lista de términos}
\printglossary[type=\acronymtype, style=altlist] % Si no esta este no imprime los acrónimos
\printglossary[style=altlist] % Si no esta este con el anterior no imprime los términos del glosario
%----------------
\end{document}

